\documentclass[a4paper,11pt]{ltjsarticle}
\usepackage[no-math]{luatexja-fontspec}
\usepackage[scale=0.9]{roboto}
\usepackage{physics2}
\usepackage{derivative}
\usepackage{bm}
\usephysicsmodule{ab}
\usepackage{amsthm}
\usepackage{amsmath}
\usepackage{amsmath}
\usepackage{amssymb}
\usepackage{amsfonts}
\usepackage{mathtools}
\NewDocumentCommand{\Set}{m o O{\middle |}}{\IfValueTF{#2}{\left\{#1\ #3\ #2\right\}}{\left\{#1\right\}}}
\NewDocumentCommand{\Coex}{m m}{E\left[#1 \ \middle | #2 \right]}
\providecommand{\floor}[1]{\ab \lfloor#1\rfloor}
\newcommand{\otherwise}{\mathrm{otherwise}}
\newcommand{\suchthat}{\mathrm{\ s.t.\ }}
\newcommand{\Number}{\mathbb{N}}
\newcommand{\Zahlen}{\mathbb{Z}}
\newcommand{\Quotient}{\mathbb{Q}}
\newcommand{\Real}{\mathbb{R}}
\newcommand{\Complex}{\mathbb{C}}
\title{統計検定(社会科学)}
\author{田淵進}
\theoremstyle{definition}
\newtheorem{remark}{Remark}
\newtheorem{definition}[remark]{Definition}
\newtheorem{theorem}[remark]{Theorem}
\DeclareMathOperator{\Cov}{Cov}
\NewDocumentCommand{\ConditionalVariance}{m m}{V\left[#1 \ \middle | #2 \right]}
\usepackage[japanese,english]{babel}
\begin{document}
\maketitle
\part{2023年}
\section{}
略
\section{}
\subsection{}
\subsection{}
一般に,価格および数量ラスパイレス指数はそれぞれ
\begin{align*}
    P_{L}&=\frac{\sum_{i}^{n}p_{1i}n_{0i}}{\sum_{i}^{n}p_{0i}n_{0i}}\times100\\
    Q_{L}&=\frac{\sum_{i}^{n}p_{0i}n_{1i}}{\sum_{i}^{n}p_{0i}n_{0i}}\times100
\end{align*}
で与えられ,パーシェ指数はそれぞれ
\begin{align*}
    P_{P}&=\frac{\sum_{i}^{n}p_{1i}n_{1i}}{\sum_{i}^{n}p_{0i}n_{1i}}\times100\\
    Q_{P}&=\frac{\sum_{i}^{n}p_{0i}n_{0i}}{\sum_{i}^{n}p_{0i}n_{1i}}\times100
\end{align*}
で与えられる.よって表1からの計算にっよって,ラスパイレス価格指数は
\begin{align*}
    P_{L}&=\frac{10363\times 43.12+2272\times 118.17}{10363\times 44.74+2272\times 119.86} \times 100\\
        &=0.9718\times 100=97.18
\end{align*}
となる.
\subsection{}
前問の定義から
\begin{align*}
    P_{L}&=\bar{x}\\
    Q_{L}&=\bar{y}
\end{align*}
である.
\subsection{}
いま,
\begin{align*}
    s_{xy}&=\sum^{n}_{n=1}w_{i}(x_{i}-\bar{x})(y_{i}-\bar{y})\\
    &=\sum^{n}_{i=1}w_{i}x_{i}y_{i}-\bar{y}\sum^{n}_{i=1}w_{i}x_{i}-\bar{x}\sum^{n}_{i=1}w_{i}y_{i}+\bar{x}\bar{y}\sum^{n}_{i=1}w_{i}\\
    &=\sum^{n}_{i=1}w_{i}x_{i}y_{i}-P_{L}Q_{L}
\end{align*}
とかける.ここで,
\begin{align*}
    \sum^{n}_{i=1}w_{i}x_{i}y_{i}&=\frac{\sum_{i}^{n}p_{1i}n_{1i}}{\sum_{i}^{n}p_{0i}n_{0i}}\\
    &=\frac{\sum_{i}^{n}p_{1i}n_{1i}}{\sum_{i}^{n}p_{0i}n_{1i}}\times\frac{\sum_{i}^{n}p_{0i}n_{1i}}{\sum_{i}^{n}p_{0i}n_{0i}}\\
    &=P_{P}Q_{L}
\end{align*}
となるので,
\begin{align*}
    s_{xy}&=P_{P}Q_{L}-P_{L}Q_{L}\\
    &=Q_{L}(P_{P}-P_{L})
\end{align*}
と示される.
\subsection{}
$P_{L}$と$Q_{L}$の大小関係は$s_{xy}$の符号と等しい.一般的に,価格と数量の間には負の相関がある場合が多いため,$P_{P}$は$P_{L}$より大きくなる傾向がある.
\newpage
\section{}
\subsection{}
$t=1,T$のみのデータを使った場合,モデルのLSEは
\begin{align*}
    Y_{t}&=\tilde{\alpha}+\tilde{\beta} X_{t}+\epsilon_{t}\quad(t=1,T)
\end{align*}
を満たすので,
\begin{align*}
    \tilde{\beta}&=\frac{Y_{T}-Y_{1}}{X_{T}-X_{1}}\\
    \tilde{\alpha}&=\frac{1}{2}(Y_{1}+Y_{T})-\tilde{\beta}\frac{1}{2}(X_{1}+X_{T})
    &=\frac{X_{1}}{X_{T}-X_{1}}Y_{1}-\frac{X_{T}}{X_{T}-X_{1}}Y_{T}
\end{align*}
となるので,
\begin{align*}
    V[\tilde{\alpha}]&=\frac{X_{1}^{2}+X_{T}^{2}}{(X_{T}-X_{1})^{2}}\sigma^{2}\\
    V[\tilde{\beta}]&=\frac{2}{(X_{T}-X_{1})^{2}}\sigma^{2}
\end{align*}
である.
\subsection{}
いま,$m=(X_{1}+X_{T})/2$とおいて
\begin{align*}
    \sum^{T}_{t=1}(X-\bar{X})^{2}=\frac{(X_{T}-X_{1})^{2}}{2}+2\pab{\bar{X}-m}^{2}+\sum^{T-1}_{t=2}(X_{t}-\bar{X})^{2}
\end{align*}
と変形できるので,条件は
\begin{align*}
    \bar{X}=m\\
    X_{2}=\cdots=X_{T-1}=m
\end{align*}
である.
\subsection{}
前問の値を用いて,
\begin{align*}
    V\bab{\tilde{Y}_{T+1}}&=V[\tilde{\alpha}+\tilde{\beta}X_{T+1}]\\
    &=V\bab{\frac{X_{T+1}-X_{T}}{X_{T}-X_{1}}Y_{1}+\frac{X_{T+1}-X_{1}}{X_{T}-X_{1}}Y_{T}}\\
    &=\frac{(X_{T+1}-X_{T})^{2}+(X_{1}-X_{T+1})^{2}}{(X_{T}-X_{1})^{2}}\sigma^{2}
\end{align*}
\subsection{}
モデルが正しいと仮定したとき,$3$点で測る際に最も$V[\tilde{\beta}]$を小さくするには,$X_{c}$を$X_{1}$または$X_{T}$と同じ値にするのが合理的である.\par
一方,単回帰であるかどうかを検証するためには,直線からのズレが大きくなる可能性が大きい$X_{c}=(X_{1}+X_{T})/2$が適当である.
\newpage
\section{}
\subsection{}
\subsection{}
$t$期と$t-4$期の関係を見て,$A,B,C$がそれぞれ$a=0.8,0,-0.8$である事がわかる.
\subsection{}
ARモデルをMAモデルとして表現すると,
\begin{align*}
    y_{t}&=ay_{t-4}+v_{t}\\
        &=a(ay_{t-8}+v_{t-4})+v_{t}\\
        &=v_{t}+av_{t-4}+a^{2}v{t-8}+a^{3}v_{t-12}+\cdots
\end{align*}
となる.あるいは,恒等作用素を$I$,ラグ作用素を$L$とおくと,$\vab{a}<1$のもとで$I-L^{4}$は可逆であり,
\begin{align*}
    y_{t}&=\pab{I-aL^{4}}^{-1}v_{t}\\
         &=(I+aL^{4}+a^{2}L^{8}+)v_{t}
\end{align*}
として,同じ結果を得る.\par
また,条件より$y{t}$は定常であるので,$\mu\coloneqq E[y_{t}]$とおくと,$\mu=a\mu$すなわち$\mu=0$であり,$\psi\coloneqq V[y_{t}]$とすると,
\begin{align*}
    \psi=&a^{2}\psi+\sigma^{2}\\
    \psi&=\frac{\sigma^{2}}{1-a^2}
\end{align*}
となる.また,共分散は
\begin{align*}
    \Cov\pab{y_{t},y_{t-k}}&=E\bab{\pab{\sum^{\infty}_{i=0}a^{i}v_{t-4i}}\pab{\sum^{\infty}_{i=0}a^{i}v_{t-k-4i}}}\\
        &=\begin{cases}
            \sum^{\infty}_{i=1} a^{2i+k/4}\sigma^{2}&(k\in 4\Zahlen)\\
            0&(\otherwise)
        \end{cases}\\
        &=\begin{cases}
            \frac{a^{k/4}\sigma^{2}}{1-a^{2}}&(k\in 4\Zahlen)\\
            0&(\otherwise)\\ 
        \end{cases}
\end{align*}
であるので,自己相関係数は,
\begin{align*}
    \rho_{k}&=\Cov\pab{y_{i},y_{i-k}}/\psi\\
            &=\begin{cases}
            a^{k/4}&(k\in 4\Zahlen)\\
            0&(\otherwise)
            \end{cases}
\end{align*}
\subsection{}
自己回帰モデルにおけるSLEは,$t=5,\ldots,T$までの$T-4$個の関係式
\begin{align*}
    y_{t}=ay_{t-4}+v_{t}
\end{align*}
から,平方和$\sum_{t=5}^{T}(y_{t}-ay_{t-4})^{2}$を最小化する値として求められる.これは通常の単回帰と同じで,SLEとして
\begin{align*}
    \hat{\alpha}=\frac{\sum^{T}_{t=5}y_{t}y_{t-4}}{\sum^{T}_{t=5}y_{t-4}^{2}}
\end{align*}
が求まる.このとき,残差の自由度は$T-5$になるので,最小化された平方和から計算される分散$\sigma^{2}$の推定量としては,
\begin{align*}
    \hat{\sigma^{2}}&=\frac{1}{T-5}\sum^{T}_{t=4}\pab{y_{t}-\hat{a}y_{t-4}}^{2}
\end{align*}
と取れる.このとき,分母にも確率項があり,一般に推定量が不偏とは限らない.\par
一方で,大数の法則より\footnote{大数の法則がちゃんと使えることを示すのはむずい気がする},
\begin{align*}
    \frac{1}{T-4}\sum^{T}_{t=5}v_{t}y_{t-4}&\to 0
    \frac{1}{T-4}\sum^{T}_{t=5}y_{t-4}y_{t-4}&\to a
\end{align*}
が成り立ち,連続写像定理から$\hat{\alpha}$は$a$に確率収束する.\par
同様に,$e_{t}\coloneqq y_{t}-\hat{a}y_{t-4}=v_{t}+(a-\hat{a})y_{t-4}$として大数の法則を用いることで,一致推定量であることが示される.
\subsection{}
$k=4(n-1)+i, i\in \Set{1,2,3,4}$とおくと,
\begin{align*}
    \hat{y}_{T+k}&=\hat{a}^{n}y_{T-4+i}
\end{align*}
と推定できる.このときの予測誤差は
\begin{align*}
    y_{T+k}-\hat{y}_{T+k}&=\pab{a^{n}-\hat{a}^{n}}y_{T-4+i}+\sum^{n}_{j=0}a^{j}v_{T+k-4j}
\end{align*}
となる.
\section{}
\subsection{}
分布の再生成より,$D\sim N(0,2\sigma^{2})$となるので,
\begin{align*}
    E[D]&=0\\
    V[D]&=2\sigma_{X}^{2}=72\\
    P(D\leq -4)&=P(Z\leq -4/\sqrt{72})&\approx0.32
\end{align*}
となる.ここで,$Z$は標準正規分布に従う確率変数とする.
\subsection{}
条件より,$Y\mid X \sim N(\alpha+\beta X,\sigma^{2})$となるので,$\beta=3/4,\alpha=120-120\beta$を用いて
\begin{align*}
    \Coex{Y}{X=132}&=\alpha+132\beta\\
    &=129
\end{align*}
となり,$V[Y\mid X=132]=\sigma^{2}=\sigma^{2}_{Y}-\sigma^{2}_{XY}/\sigma^{2}_{X}=63$である.\par
ここで,平均への回帰効果とは,母平均が同じであるとすれば,一回目で132を記録しても,2回目の測定値の期待値は132よりも小さい値になることを指しており,平均への回帰分は3mmHgとみなされる.
\subsection{}
Tower Propertyなどから,
\begin{align*}
    E\bab{(X-\mu)^{2}}&=E\bab{\Coex{(X-\mu)^{2}}{\theta}}\\
    &=E\bab{\Coex{(X-\theta)^{2}+(\theta-\mu)^{2}+2(X-\mu)(X-\theta)}{\theta}}\\
    &=\psi^{2}+\tau^{2}
\end{align*}
と計算できる.同様に$V[Y]=\psi^{2}+\tau^{2}$であり,$\Cov(X,Y)=\tau^{2}$と計算できる.これに$\psi^{2}+\tau^{2}=144, \tau^{2}=3(\psi^{2}+\tau^{2})/4$を代入して,
\begin{align*}
    \tau^{2}=108\\
    \psi^{2}=36
\end{align*}
を得る.
\subsection{}
条件から,$\theta$と$\varepsilon$はそれぞれ独立に正規分布に従うので,
\begin{align*}
    \begin{pmatrix}
        \theta\\
        X
    \end{pmatrix}
    &=\begin{pmatrix}
     1&0\\
     1&1   
    \end{pmatrix}
    \begin{pmatrix}
        \theta\\
        \varepsilon_{1}
    \end{pmatrix}\sim N\pab{
        \begin{pmatrix}
            \mu\\
            \mu
        \end{pmatrix},\begin{pmatrix}
     1&1\\
     0&1   
    \end{pmatrix}
    \begin{pmatrix}
     \tau^{2}&0\\
     0&\psi^{2}   
    \end{pmatrix}
    \begin{pmatrix}
     1&0\\
     1&1   
    \end{pmatrix}
    }
\end{align*}
となる.特に共分散行列は
\begin{align*}
    \begin{pmatrix}
     1&1\\
     0&1   
    \end{pmatrix}
    \begin{pmatrix}
     \tau^{2}&0\\
     0&\psi^{2}   
    \end{pmatrix}
    \begin{pmatrix}
     1&0\\
     1&1   
    \end{pmatrix}
    =    \begin{pmatrix}
     \tau^{2}+\psi^{2}&\psi^{2}\\
     \psi^{2}&\psi^{2}   
    \end{pmatrix}
\end{align*}
である.このとき,$\theta\mid X$の分布は,$X=132$とすると
\begin{align*}
    \Coex{\theta}{X}&=\mu+\frac{\Cov(\theta,X)}{V[X]}(X-\mu)\\
    &=129\\
    \ConditionalVariance{\theta}{X}&=V[\theta]-\frac{\Cov(\theta,X)^{2}}{V[X]}\\
    &=27
\end{align*}
に従う正規分布となる.
\subsection{}
$\varepsilon_{1},\varepsilon_{2}$は互いに独立であるので,$\theta+\varepsilon_{1}\mid X$の分布は$N(129,63)$となる.よって,
\begin{align*}
    P(Y\leq 128\mid X=132)=P(Z\leq -1/\sqrt{63})\approx 0.45
\end{align*}
となる.ここで$Z$は標準正規分布に従う確率変数とする.
\end{document}