\documentclass[a4paper,11pt]{ltjsarticle}
\usepackage[no-math]{luatexja-fontspec}
\usepackage[scale=0.9]{roboto}
\usepackage{physics2}
\usepackage{derivative}
\usepackage{bm}
\usephysicsmodule{ab}
\usepackage{amsthm}
\usepackage{amsmath}
\usepackage{amssymb}
\usepackage{amsfonts}
\usepackage{mathtools}
\NewDocumentCommand{\Set}{m o O{\middle |}}{\IfValueTF{#2}{\left\{#1\ #3\ #2\right\}}{\left\{#1\right\}}}
\NewDocumentCommand{\Coex}{m m}{E\left[#1 \ \middle | #2 \right]}
\providecommand{\floor}[1]{\ab \lfloor#1\rfloor}
\newcommand{\otherwise}{\mathrm{otherwise}}
\newcommand{\suchthat}{\mathrm{\ s.t.\ }}
\newcommand{\Number}{\mathbb{N}}
\newcommand{\Zahlen}{\mathbb{Z}}
\newcommand{\Quotient}{\mathbb{Q}}
\newcommand{\Real}{\mathbb{R}}
\newcommand{\Complex}{\mathbb{C}}
\title{統計検定(社会科学)}
\author{田淵進}
\theoremstyle{definition}
\newtheorem{remark}{Remark}
\newtheorem{definition}[remark]{Definition}
\newtheorem{theorem}[remark]{Theorem}
\usepackage[japanese,english]{babel}
\begin{document}
\maketitle
\part{2023年}
\section{}
略
\section{}
\subsection{}
\subsection{}
一般に、価格および数量ラスパイレス指数はそれぞれ
\begin{align*}
    P_{L}&=\frac{\sum_{i}^{n}p_{1i}n_{0i}}{\sum_{i}^{n}p_{0i}n_{0i}}\times100\\
    Q_{L}&=\frac{\sum_{i}^{n}p_{0i}n_{1i}}{\sum_{i}^{n}p_{0i}n_{0i}}\times100
\end{align*}
で与えられ、パーシェ指数はそれぞれ
\begin{align*}
    P_{P}&=\frac{\sum_{i}^{n}p_{1i}n_{1i}}{\sum_{i}^{n}p_{0i}n_{1i}}\times100\\
    Q_{P}&=\frac{\sum_{i}^{n}p_{0i}n_{0i}}{\sum_{i}^{n}p_{0i}n_{1i}}\times100
\end{align*}
で与えられる。よって表1からの計算にっよって、ラスパイレス価格指数は
\begin{align*}
    P_{L}&=\frac{10363\times 43.12+2272\times 118.17}{10363\times 44.74+2272\times 119.86} \times 100\\
        &=0.9718\times 100=97.18
\end{align*}
となる。
\subsection{}
前問の定義から
\begin{align*}
    P_{L}&=\bar{x}\\
    Q_{L}&=\bar{y}
\end{align*}
である。
\subsection{}
いま、
\begin{align*}
    s_{xy}&=\sum^{n}_{n=1}w_{i}(x_{i}-\bar{x})(y_{i}-\bar{y})\\
    &=\sum^{n}_{i=1}w_{i}x_{i}y_{i}-\bar{y}\sum^{n}_{i=1}w_{i}x_{i}-\bar{x}\sum^{n}_{i=1}w_{i}y_{i}+\bar{x}\bar{y}\sum^{n}_{i=1}w_{i}\\
    &=\sum^{n}_{i=1}w_{i}x_{i}y_{i}-P_{L}Q_{L}
\end{align*}
とかける。ここで、
\begin{align*}
    \sum^{n}_{i=1}w_{i}x_{i}y_{i}&=\frac{\sum_{i}^{n}p_{1i}n_{1i}}{\sum_{i}^{n}p_{0i}n_{0i}}\\
    &=\frac{\sum_{i}^{n}p_{1i}n_{1i}}{\sum_{i}^{n}p_{0i}n_{1i}}\times\frac{\sum_{i}^{n}p_{0i}n_{1i}}{\sum_{i}^{n}p_{0i}n_{0i}}\\
    &=P_{P}Q_{L}
\end{align*}
となるので、
\begin{align*}
    s_{xy}&=P_{P}Q_{L}-P_{L}Q_{L}\\
    &=Q_{L}(P_{P}-P_{L})
\end{align*}
と示される。
\subsection{}
$P_{L}$と$Q_{L}$の大小関係は$s_{xy}$の符号と等しい。一般的に、価格と数量の間には負の相関がある場合が多いため、$P_{P}$は$P_{L}$より大きくなる傾向がある。
\newpage
\section{}
\subsection{}
$t=1,T$のみのデータを使った場合、モデルのLSEは
\begin{align*}
    Y_{t}&=\tilde{\alpha}+\tilde{\beta} X_{t}+\epsilon_{t}\quad(t=1,T)
\end{align*}
を満たすので、
\begin{align*}
    \tilde{\beta}&=\frac{Y_{T}-Y_{1}}{X_{T}-X_{1}}\\
    \tilde{\alpha}&=\frac{1}{2}(Y_{1}+Y_{T})-\tilde{\beta}\frac{1}{2}(X_{1}+X_{T})
    &=\frac{X_{1}}{X_{T}-X_{1}}Y_{1}-\frac{X_{T}}{X_{T}-X_{1}}Y_{T}
\end{align*}
となるので、
\begin{align*}
    V[\tilde{\alpha}]&=\frac{X_{1}^{2}+X_{T}^{2}}{(X_{T}-X_{1})^{2}}\sigma^{2}\\
    V[\tilde{\beta}]&=\frac{2}{(X_{T}-X_{1})^{2}}\sigma^{2}
\end{align*}
である。
\subsection{}
いま、$m=(X_{1}+X_{T})/2$とおいて
\begin{align*}
    \sum^{T}_{t=1}(X-\bar{X})^{2}=\frac{(X_{T}-X_{1})^{2}}{2}+2\pab{\bar{X}-m}^{2}+\sum^{T-1}_{t=2}(X_{t}-\bar{X})^{2}
\end{align*}
と変形できるので、条件は
\begin{align*}
    \bar{X}=m\\
    X_{2}=\cdots=X_{T-1}=m
\end{align*}
である。
\subsection{}
前問の値を用いて、
\begin{align*}
    V[\tilde{Y}_{T+1}]&=V[\tilde{\alpha}+\tilde{\beta}X_{T+1}]\\
    &=V\bab{\frac{X_{T+1}-X_{T}}{X_{T}-X_{1}}Y_{1}+\frac{X_{T+1}-X_{1}}{X_{T}-X_{1}}Y_{T}}\\
    &=\frac{(X_{T+1}-X_{T})^{2}+(X_{1}-X_{T+1})^{2}}{(X_{T}-X_{1})^{2}}\sigma^{2}
\end{align*}
\subsection{}
モデルが正しいと仮定したとき、$3$点で測る際に最も$V[\tilde{\beta}]$を小さくするには、$X_{c}$を$X_{1}$または$X_{T}$と同じ値にするのが合理的である。\par
一方、単回帰であるかどうかを検証するためには、直線からのズレが大きくなる可能性が大きい$X_{c}=(X_{1}+X_{T})/2$が適当である。
\end{document}