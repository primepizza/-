\documentclass[a4paper,11pt]{ltjsarticle}
\usepackage[no-math]{luatexja-fontspec}
\usepackage[scale=0.9]{roboto}
\usepackage{physics2}
\usepackage{derivative}
\usepackage{bm}
\usephysicsmodule{ab}
\usepackage{amsthm}
\usepackage{amsmath}
\usepackage{amsmath}
\usepackage{amssymb}
\usepackage{amsfonts}
\usepackage{mathtools}
\NewDocumentCommand{\Set}{m o O{\middle |}}{\IfValueTF{#2}{\left\{#1\ #3\ #2\right\}}{\left\{#1\right\}}}
\NewDocumentCommand{\Coex}{m m}{E\left[#1 \ \middle | #2 \right]}
\providecommand{\floor}[1]{\ab \lfloor#1\rfloor}
\newcommand{\otherwise}{\mathrm{otherwise}}
\newcommand{\suchthat}{\mathrm{\ s.t.\ }}
\newcommand{\Number}{\mathbb{N}}
\newcommand{\Zahlen}{\mathbb{Z}}
\newcommand{\Quotient}{\mathbb{Q}}
\newcommand{\Real}{\mathbb{R}}
\newcommand{\Complex}{\mathbb{C}}
\title{統計検定(社会科学)}
\author{田淵進}
\theoremstyle{definition}
\newtheorem{remark}{Remark}
\newtheorem{definition}[remark]{Definition}
\newtheorem{theorem}[remark]{Theorem}
\DeclareMathOperator{\Cov}{Cov}
\NewDocumentCommand{\ConditionalVariance}{m m}{V\left[#1 \ \middle | #2 \right]}
\usepackage[japanese,english]{babel}
\begin{document}
\maketitle
\part{2022年}
\section{}
略
\section{}
\subsection{}
\subsection{}
一般に,価格および数量ラスパイレス指数はそれぞれ
\begin{align*}
    P_{L}&=\frac{\sum_{i}^{n}p_{1i}n_{0i}}{\sum_{i}^{n}p_{0i}n_{0i}}\times100\\
    Q_{L}&=\frac{\sum_{i}^{n}p_{0i}n_{1i}}{\sum_{i}^{n}p_{0i}n_{0i}}\times100
\end{align*}
で与えられ,パーシェ指数はそれぞれ
\begin{align*}
    P_{P}&=\frac{\sum_{i}^{n}p_{1i}n_{1i}}{\sum_{i}^{n}p_{0i}n_{1i}}\times100\\
    Q_{P}&=\frac{\sum_{i}^{n}p_{0i}n_{0i}}{\sum_{i}^{n}p_{0i}n_{1i}}\times100
\end{align*}
で与えられる.よって表1からの計算にっよって,ラスパイレス価格指数は
\begin{align*}
    P_{L}&=\frac{10363\times 43.12+2272\times 118.17}{10363\times 44.74+2272\times 119.86} \times 100\\
        &=0.9718\times 100=97.18
\end{align*}
となる.
\subsection{}
前問の定義から
\begin{align*}
    P_{L}&=\bar{x}\\
    Q_{L}&=\bar{y}
\end{align*}
である.
\subsection{}
いま,
\begin{align*}
    s_{xy}&=\sum^{n}_{n=1}w_{i}(x_{i}-\bar{x})(y_{i}-\bar{y})\\
    &=\sum^{n}_{i=1}w_{i}x_{i}y_{i}-\bar{y}\sum^{n}_{i=1}w_{i}x_{i}-\bar{x}\sum^{n}_{i=1}w_{i}y_{i}+\bar{x}\bar{y}\sum^{n}_{i=1}w_{i}\\
    &=\sum^{n}_{i=1}w_{i}x_{i}y_{i}-P_{L}Q_{L}
\end{align*}
とかける.ここで,
\begin{align*}
    \sum^{n}_{i=1}w_{i}x_{i}y_{i}&=\frac{\sum_{i}^{n}p_{1i}n_{1i}}{\sum_{i}^{n}p_{0i}n_{0i}}\\
    &=\frac{\sum_{i}^{n}p_{1i}n_{1i}}{\sum_{i}^{n}p_{0i}n_{1i}}\times\frac{\sum_{i}^{n}p_{0i}n_{1i}}{\sum_{i}^{n}p_{0i}n_{0i}}\\
    &=P_{P}Q_{L}
\end{align*}
となるので,
\begin{align*}
    s_{xy}&=P_{P}Q_{L}-P_{L}Q_{L}\\
    &=Q_{L}(P_{P}-P_{L})
\end{align*}
と示される.
\subsection{}
$P_{L}$と$Q_{L}$の大小関係は$s_{xy}$の符号と等しい.一般的に,価格と数量の間には負の相関がある場合が多いため,$P_{P}$は$P_{L}$より大きくなる傾向がある.
\newpage
\section{}
\subsection{}
$t=1,T$のみのデータを使った場合,モデルのLSEは
\begin{align*}
    Y_{t}&=\tilde{\alpha}+\tilde{\beta} X_{t}+\epsilon_{t}\quad(t=1,T)
\end{align*}
を満たすので,
\begin{align*}
    \tilde{\beta}&=\frac{Y_{T}-Y_{1}}{X_{T}-X_{1}}\\
    \tilde{\alpha}&=\frac{1}{2}(Y_{1}+Y_{T})-\tilde{\beta}\frac{1}{2}(X_{1}+X_{T})
    &=\frac{X_{1}}{X_{T}-X_{1}}Y_{1}-\frac{X_{T}}{X_{T}-X_{1}}Y_{T}
\end{align*}
となるので,
\begin{align*}
    V[\tilde{\alpha}]&=\frac{X_{1}^{2}+X_{T}^{2}}{(X_{T}-X_{1})^{2}}\sigma^{2}\\
    V[\tilde{\beta}]&=\frac{2}{(X_{T}-X_{1})^{2}}\sigma^{2}
\end{align*}
である.
\subsection{}
いま,$m=(X_{1}+X_{T})/2$とおいて
\begin{align*}
    \sum^{T}_{t=1}(X-\bar{X})^{2}=\frac{(X_{T}-X_{1})^{2}}{2}+2\pab{\bar{X}-m}^{2}+\sum^{T-1}_{t=2}(X_{t}-\bar{X})^{2}
\end{align*}
と変形できるので,条件は
\begin{align*}
    \bar{X}=m\\
    X_{2}=\cdots=X_{T-1}=m
\end{align*}
である.
\subsection{}
前問の値を用いて,
\begin{align*}
    V\bab{\tilde{Y}_{T+1}}&=V[\tilde{\alpha}+\tilde{\beta}X_{T+1}]\\
    &=V\bab{\frac{X_{T+1}-X_{T}}{X_{T}-X_{1}}Y_{1}+\frac{X_{T+1}-X_{1}}{X_{T}-X_{1}}Y_{T}}\\
    &=\frac{(X_{T+1}-X_{T})^{2}+(X_{1}-X_{T+1})^{2}}{(X_{T}-X_{1})^{2}}\sigma^{2}
\end{align*}
\subsection{}
モデルが正しいと仮定したとき,$3$点で測る際に最も$V[\tilde{\beta}]$を小さくするには,$X_{c}$を$X_{1}$または$X_{T}$と同じ値にするのが合理的である.\par
一方,単回帰であるかどうかを検証するためには,直線からのズレが大きくなる可能性が大きい$X_{c}=(X_{1}+X_{T})/2$が適当である.
\newpage
\section{}
\subsection{}
\subsection{}
$t$期と$t-4$期の関係を見て,$A,B,C$がそれぞれ$a=0.8,0,-0.8$である事がわかる.
\subsection{}
ARモデルをMAモデルとして表現すると,
\begin{align*}
    y_{t}&=ay_{t-4}+v_{t}\\
        &=a(ay_{t-8}+v_{t-4})+v_{t}\\
        &=v_{t}+av_{t-4}+a^{2}v{t-8}+a^{3}v_{t-12}+\cdots
\end{align*}
となる.あるいは,恒等作用素を$I$,ラグ作用素を$L$とおくと,$\vab{a}<1$のもとで$I-L^{4}$は可逆であり,
\begin{align*}
    y_{t}&=\pab{I-aL^{4}}^{-1}v_{t}\\
         &=(I+aL^{4}+a^{2}L^{8}+)v_{t}
\end{align*}
として,同じ結果を得る.\par
また,条件より$y{t}$は定常であるので,$\mu\coloneqq E[y_{t}]$とおくと,$\mu=a\mu$すなわち$\mu=0$であり,$\psi\coloneqq V[y_{t}]$とすると,
\begin{align*}
    \psi=&a^{2}\psi+\sigma^{2}\\
    \psi&=\frac{\sigma^{2}}{1-a^2}
\end{align*}
となる.また,共分散は
\begin{align*}
    \Cov\pab{y_{t},y_{t-k}}&=E\bab{\pab{\sum^{\infty}_{i=0}a^{i}v_{t-4i}}\pab{\sum^{\infty}_{i=0}a^{i}v_{t-k-4i}}}\\
        &=\begin{cases}
            \sum^{\infty}_{i=1} a^{2i+k/4}\sigma^{2}&(k\in 4\Zahlen)\\
            0&(\otherwise)
        \end{cases}\\
        &=\begin{cases}
            \frac{a^{k/4}\sigma^{2}}{1-a^{2}}&(k\in 4\Zahlen)\\
            0&(\otherwise)\\ 
        \end{cases}
\end{align*}
であるので,自己相関係数は,
\begin{align*}
    \rho_{k}&=\Cov\pab{y_{i},y_{i-k}}/\psi\\
            &=\begin{cases}
            a^{k/4}&(k\in 4\Zahlen)\\
            0&(\otherwise)
            \end{cases}
\end{align*}
\subsection{}
自己回帰モデルにおけるSLEは,$t=5,\ldots,T$までの$T-4$個の関係式
\begin{align*}
    y_{t}=ay_{t-4}+v_{t}
\end{align*}
から,平方和$\sum_{t=5}^{T}(y_{t}-ay_{t-4})^{2}$を最小化する値として求められる.これは通常の単回帰と同じで,SLEとして
\begin{align*}
    \hat{\alpha}=\frac{\sum^{T}_{t=5}y_{t}y_{t-4}}{\sum^{T}_{t=5}y_{t-4}^{2}}
\end{align*}
が求まる.このとき,残差の自由度は$T-5$になるので,最小化された平方和から計算される分散$\sigma^{2}$の推定量としては,
\begin{align*}
    \hat{\sigma^{2}}&=\frac{1}{T-5}\sum^{T}_{t=4}\pab{y_{t}-\hat{a}y_{t-4}}^{2}
\end{align*}
と取れる.このとき,分母にも確率項があり,一般に推定量が不偏とは限らない.\par
一方で,大数の法則より\footnote{大数の法則がちゃんと使えることを示すのはむずい気がする},
\begin{align*}
    \frac{1}{T-4}\sum^{T}_{t=5}v_{t}y_{t-4}&\to 0
    \frac{1}{T-4}\sum^{T}_{t=5}y_{t-4}y_{t-4}&\to a
\end{align*}
が成り立ち,連続写像定理から$\hat{\alpha}$は$a$に確率収束する.\par
同様に,$e_{t}\coloneqq y_{t}-\hat{a}y_{t-4}=v_{t}+(a-\hat{a})y_{t-4}$として大数の法則を用いることで,一致推定量であることが示される.
\subsection{}
$k=4(n-1)+i, i\in \Set{1,2,3,4}$とおくと,
\begin{align*}
    \hat{y}_{T+k}&=\hat{a}^{n}y_{T-4+i}
\end{align*}
と推定できる.このときの予測誤差は
\begin{align*}
    y_{T+k}-\hat{y}_{T+k}&=\pab{a^{n}-\hat{a}^{n}}y_{T-4+i}+\sum^{n}_{j=0}a^{j}v_{T+k-4j}
\end{align*}
となる.
\section{}
\subsection{}
分布の再生成より,$D\sim N(0,2\sigma^{2})$となるので,
\begin{align*}
    E[D]&=0\\
    V[D]&=2\sigma_{X}^{2}=72\\
    P(D\leq -4)&=P(Z\leq -4/\sqrt{72})&\approx0.32
\end{align*}
となる.ここで,$Z$は標準正規分布に従う確率変数とする.
\subsection{}
条件より,$Y\mid X \sim N(\alpha+\beta X,\sigma^{2})$となるので,$\beta=3/4,\alpha=120-120\beta$を用いて
\begin{align*}
    \Coex{Y}{X=132}&=\alpha+132\beta\\
    &=129
\end{align*}
となり,$V[Y\mid X=132]=\sigma^{2}=\sigma^{2}_{Y}-\sigma^{2}_{XY}/\sigma^{2}_{X}=63$である.\par
ここで,平均への回帰効果とは,母平均が同じであるとすれば,一回目で132を記録しても,2回目の測定値の期待値は132よりも小さい値になることを指しており,平均への回帰分は3mmHgとみなされる.
\subsection{}
Tower Propertyなどから,
\begin{align*}
    E\bab{(X-\mu)^{2}}&=E\bab{\Coex{(X-\mu)^{2}}{\theta}}\\
    &=E\bab{\Coex{(X-\theta)^{2}+(\theta-\mu)^{2}+2(X-\mu)(X-\theta)}{\theta}}\\
    &=\psi^{2}+\tau^{2}
\end{align*}
と計算できる.同様に$V[Y]=\psi^{2}+\tau^{2}$であり,$\Cov(X,Y)=\tau^{2}$と計算できる.これに$\psi^{2}+\tau^{2}=144, \tau^{2}=3(\psi^{2}+\tau^{2})/4$を代入して,
\begin{align*}
    \tau^{2}=108\\
    \psi^{2}=36
\end{align*}
を得る.
\subsection{}
条件から,$\theta$と$\varepsilon$はそれぞれ独立に正規分布に従うので,
\begin{align*}
    \begin{pmatrix}
        \theta\\
        X
    \end{pmatrix}
    &=\begin{pmatrix}
     1&0\\
     1&1   
    \end{pmatrix}
    \begin{pmatrix}
        \theta\\
        \varepsilon_{1}
    \end{pmatrix}\sim N\pab{
        \begin{pmatrix}
            \mu\\
            \mu
        \end{pmatrix},\begin{pmatrix}
     1&1\\
     0&1   
    \end{pmatrix}
    \begin{pmatrix}
     \tau^{2}&0\\
     0&\psi^{2}   
    \end{pmatrix}
    \begin{pmatrix}
     1&0\\
     1&1   
    \end{pmatrix}
    }
\end{align*}
となる.特に共分散行列は
\begin{align*}
    \begin{pmatrix}
     1&1\\
     0&1   
    \end{pmatrix}
    \begin{pmatrix}
     \tau^{2}&0\\
     0&\psi^{2}   
    \end{pmatrix}
    \begin{pmatrix}
     1&0\\
     1&1   
    \end{pmatrix}
    =    \begin{pmatrix}
     \tau^{2}+\psi^{2}&\psi^{2}\\
     \psi^{2}&\psi^{2}   
    \end{pmatrix}
\end{align*}
である.このとき,$\theta\mid X$の分布は,$X=132$とすると
\begin{align*}
    \Coex{\theta}{X}&=\mu+\frac{\Cov(\theta,X)}{V[X]}(X-\mu)\\
    &=129\\
    \ConditionalVariance{\theta}{X}&=V[\theta]-\frac{\Cov(\theta,X)^{2}}{V[X]}\\
    &=27
\end{align*}
に従う正規分布となる.
\subsection{}
$\varepsilon_{1},\varepsilon_{2}$は互いに独立であるので,$\theta+\varepsilon_{1}\mid X$の分布は$N(129,63)$となる.よって,
\begin{align*}
    P(Y\leq 128\mid X=132)=P(Z\leq -1/\sqrt{63})\approx 0.45
\end{align*}
となる.ここで$Z$は標準正規分布に従う確率変数とする.
\newpage
\part{2023年}
\setcounter{section}{0}
\section{}
\subsection{}
いま,与えられた条件から
\begin{align*}
    \bar{x}&\approx3.7327\\
    \bar{y}&\approx27.49\\
    \bar{x^{2}}&\approx 16.15\\
    \bar{y^{2}}&\approx 855.27\\
    \bar{xy}&\approx 108.95
\end{align*}
よって,$n=33$であることから,不偏分散は
\begin{align*}
    s_{x}^{2}&=\frac{33}{32}\pab{\bar{x^{2}}-\bar{x}^{2}}\approx 2.33\\
    s_{y}^{2}&=\frac{33}{32}\pab{\bar{y^{2}}-\bar{y}^{2}}\approx \frac{33}{32}\pab{855.27 - 27.49^{2}} \approx 102.68\\
    s_{xy}&=\frac{33}{32}\pab{\bar{xy}-\bar{x}\bar{y}}\approx \frac{33}{32}\pab{108.95 - 3.7327\times 27.49} \approx 6.54
\end{align*}
また,相関係数は
\begin{align*}
    r_{xy}&=\frac{s_{xy}}{\sqrt{s_{x}^{2}s_{y}^{2}}}\approx \frac{6.54}{\sqrt{2.33\times 102.68}}\approx 0.42
\end{align*}
\subsection{}
デルタ法とは,観測値の関数$g(t_{1},\ldots,t_{n})$で表される統計量の分散を,$g$の各変数の平均値周りでの1次近似によって求める手法である.いま,
\begin{align*}
    \hat{R}=g(\bar{x},\bar{y})&\coloneqq \frac{\bar{y}}{\bar{x}}\\
    &\approx g(\mu_{X},\mu_{X})+\pdv{g(\mu_{X},\mu_{X})}{\mu_{X}}(\bar{x}-\mu_{X})+\pdv{g(\mu_{X},\mu_{Y})}{\mu_{Y}}(\bar{y}-\mu_{Y})\\
    &=\frac{\mu_{Y}}{\mu_{X}}-\frac{\mu_{Y}}{\mu_{X}^{2}}(\bar{x}-\mu_{X})+\frac{1}{\mu_{X}}(\bar{y}-\mu_{Y})
\end{align*}
と近似されるので,求める分散は
\begin{align*}
    V\bab{\hat{R}}&=\frac{1}{\mu_{X}^{2}}V\bab{\bar{y}-R\bar{x}}\\
    &=\frac{1}{n\mu_{X}^{2}}\pab{E\bab{(Y-RX)^{2}}-0}
\end{align*}
となる.
\subsection{}
前問の結果を用いて
\begin{align*}
    V\bab{\hat{\mu_{Y}}}&=\mu^{2}_{X}V[\hat{R}]\\
        &=\frac{1}{n}\pab{V[Y]+R^{2}V[X]-2R\Cov\pab{X,Y}}\\
        &=\frac{\mu^{2}_{Y}}{n}\pab{\frac{\sigma^{2}_{Y}}{\mu_{Y}^{2}}+\frac{\sigma_{X}^{2}}{\mu^{2}_{Y}}\frac{\mu^{2}_{Y}}{\mu^{2}_{X}}-2\frac{\Cov{(X,Y)}}{\mu_{X}\mu_{Y}}}\\
        &=\frac{\mu^{2}_{Y}}{n}\pab{\mathrm{cv}(Y)^{2}+\mathrm{cv}(X)^{2}-2\rho\mathrm{cv}(X)\mathrm{cv}(Y)}
\end{align*}
と求まる.
\subsection{}
$\hat{R}=\bar{y}/\mu_{X}$としたときの分散は$V_{1}\coloneqq \sigma^{2}_{Y}/\mu^{2}_{X}$である.一方,$\bar{y}/\bar{x}$とした場合の分散は,前問での計算から
\begin{align*}
    V\bab{\hat{R}}\coloneqq V_{2}= \frac{1}{n\mu_{X}^{2}}\pab{V[Y]+R^{2}V[X]-2R\Cov(X,Y)}
\end{align*}
となるので,$V_{1}>V_{2}$となるのは
\begin{align*}
    R\sigma^{2}_{X}&<2\frac{\sigma_{XY}}{\sigma_{X}\sigma_{Y}}
\end{align*}
これを変形すると、
\begin{align*}
    \frac{\mathrm{cv}(X)}{2\mathrm{cv}(Y)}&<\rho
\end{align*}
となるときである.よって,観測値から求まる$\mathrm{cv}(X),\mathrm{cv}(Y),\rho$を参考に,どちらを使うか切り替えるのが望ましい.
\newpage
\section{}
\subsection{}
$X$がパレート分布$(a,b)$に従うとき,
\begin{align*}
    E[X]&=\int^{\infty}_{n}\frac{ab^{a}x}{x^{a+1}}dx\\
    &=ab^{a}\bab{\frac{-1}{(a-1)x^{a-1}}}^{\infty}_{b}
    &=\frac{ab}{a-1}
\end{align*}
となり,累積分布関数は
\begin{align*}
    F(x)&=\int^{x}_{b}\frac{ab^{a}}{t^{a+1}}dt\\
    &=\bab{\frac{-b^{a}}{t^{a}}}^{x}_{b}\\
    &=1-\pab{\frac{b}{t}}^{a}
\end{align*}
となる.よって中央値は$m=b\sqrt[a]{2}$である.
\subsection{}
$c>b$として,条件付き累積分布を考えると,
\begin{align*}
    F(x\mid x>c)&=P(X\leq x\mid x>c)\\
    &=\frac{P(c<X\leq x)}{P(X>c)}\\
    &=\frac{\pab{F(x)-F(c)}}{F(c)}\\
    &=\pab{\frac{c}{b}}^{a}\pab{\pab{\frac{b}{c}}^{a}-\pab{\frac{b}{x}}^{a}}\\
    &=1-\pab{\frac{c}{x}}
\end{align*}
となるので,$X\mid X>c$はパラメータ$(a,c)$のパレート分布に従う.よって,下限で条件づけたパレート分布は再びパレート分布に従うことがわかる.
\subsection{}
パレート分布における80:20の法則とは,上位2割の領域における期待値が,全体の期待値の8割を占める状態だと考えられる.すなわち,
\begin{align*}
    P(X>x_0)=0.2
\end{align*}
となるような$x_{0}$に対して,
\begin{align*}
    E[X,X>x_{0}]=\int^{\infty}_{x_{0}}\frac{ab^{a}}{x^{a}}dx=0.8E[X]
\end{align*}
が満たされれば良い.いま,$b=1$
\begin{align*}
    P(X>x_0)&=\pab{\frac{1}{x_{0}}}^{a}=0.2\\
    E[X,X>x_{0}]&=\pab{\frac{1}{x_{0}}}^{a-1}=0.8
\end{align*}
なので,$x_{0}=4$であり,$a=\log{5}/\log{4}$\\
\subsection{}
$b=1,a=\log{5}/\log{4}$のとき,上位20\%に入るには前問の結果から$x=4$(百万)である.また,上位4\%に入るには,上位20\%の中で更に上位20\%のに入ればよいので,前問の結果より,$b=4$と取り直すことで$x=16$(百万)が得られる.
\newpage
\section{}
\subsection{}
条件より,
\begin{align*}
    E[\exp(kU)]&=\frac{1}{\sqrt{2\pi\sigma^{2}_{u}}}\int_{\Real}\exp\pab{\frac{-u^{2}}{2\sigma^{2}_{u}}+ku}du\\
    &=\frac{1}{\sqrt{2\pi\sigma^{2}_{u}}}\int_{\Real}\exp\pab{\frac{-u^{2}+2ku-k^{2}}{2\sigma^{2}_{u}}+\frac{k^{2}\sigma^{2}_{u}}{2}}du\\
    &=\exp\pab{\frac{k^{2}}{2}\sigma^{2}_{u}}
\end{align*}
\subsection{}
$Y_{t}=e^{\alpha_{0}}Y^{\alpha_{1}}_{t-1}\exp(U_{t})$であることに注意すると,
\begin{align*}
    \Coex{Y_{t}}{Y_{t-1},\ldots,Y_{1}}&=\Coex{Y_{t}}{Y_{t-1}}\\
    &=e^{\alpha_{0}}Y^{\alpha_{1}}_{t-1}E\bab{\exp(U_{t})}
    &=\exp\pab{\alpha_{0}+\sigma^{2}_{u}/2}Y^{\alpha_{1}}_{t-1}
\end{align*}
となるので,$\alpha_{0}=-\sigma^{2}/2,\alpha_{1}=1$がマルチンゲールとなるための条件である.
\subsection{}
$T$までの値をもとに,$Y_{T+1}$の値を予測すると,予測分散は,
\begin{align*}
    V\bab{\tilde{Y}_{T+1}}&=V\bab{e^{\alpha_{0}}Y^{\alpha_{1}}_{T}\exp(U_{T+1})}\\
    &=\exp\pab{-\sigma^{2}}Y^{2}_{T}\pab{E[\exp(U_{T+1}^{2})]-E[\exp(U_{T+1})]^{2}}\\
    &=Y_{T}^{2}\pab{e^{\sigma^{2}_{u}}-1}\\
\end{align*}
となる.
\subsection{}
通常,回帰分析では最小二乗法によって推定量$\hat{\alpha_{1}}$を求め,中心極限定理より,$\sqrt{\hat{\alpha_{1}}-\alpha_1}$が正規分布に近似できることを利用して検定を行う.\par
このとき,$\vab{\alpha_{1}}<1$においては,極限分布は$N(1,1-\alpha^{2})$となるが,$\alpha_{1}=1$においては正規分布に収束しないため,通常の$t$検定で検定することはできない.
\newpage
\section{}
\subsection{}
表から回帰係数を求めればいい.適当な二つの場合を持ってきて,男性の場合は
\begin{align*}
    7.2&=a+5b\\
    7.8&=a+6b
\end{align*}
となるので,$a=4.2,b=0.6$となり,女性の場合は
\begin{align*}
    6.6&=a+5b\\
    7.2&=a+6b\\
\end{align*}
となるので,$a=3.6,b=0.6$となる.よって,男性の方が同じ時間を働いたときの賃金が高いといえる.
\subsection{}
いま,労働時間と賃金の平均値を考える.同じ労働時間での賃金の平均値は予測値と一致することに注意すると,
\begin{align*}
    \bar{x}_{M}&=5\times 0.1+6\times0.15+7\times 0.4 +8\times 0.35=7\\
    \bar{y}_{M}=\bar{\hat{y}}_{M}&=7.2\times 0.1+7.8\times0.15+8.4\times 0.4 +9.0\times 0.35=8.4\\
    \bar{x}_{F}&=5\times 0.35+6\times0.4+7\times 0.15 +8\times 0.1=6\\
    \bar{y}_{F}=\bar{\hat{y}}_{F}&=6.6\times 0.35+7.2\times0.4+7.8\times 0.15 +8.4\times 0.1 =7.2
\end{align*}
よって$y_{M}/x_{M}=y_{F}/x_{F}=1.2$となり,男女で等しくなる.
\subsection{}
まず,$y$の分散を求める必要がある.いま,各データ$(x_{i},y_{i})$は
\begin{align*}
    y_{i}=a+bx_{i}+u_{i}
\end{align*}
と表されるとする.ここで,$\hat{y}_{i}=a+bx_{i}$とおいて,全体のデータ数を$N$とすると,最小二乗法では
\begin{align*}
    Ns_{e}^{2}=\sum_{i=1}^{N}\pab{y_{i}-\tilde{y}_{i}}^{2}
\end{align*}
が最小化されるように$a,b$を決定する.ここで,$x_{i}$が固定されると$\tilde{y}_{i}$の値が決定されることに注意すると,
\begin{align*}
    Ns_{e}^{2}&=\sum\pab{y_{i}-\bar{y}+\bar{y}-\tilde{y}_{i}}^{2}\\
    &=N\pab{V[y]+V[\tilde{y}]}-2\sum_{j=5,\ldots,8}\pab{\tilde{y}_{i}-\bar{y}}\sum_{x_{i}\equiv j}\pab{y_{i}-\bar{y}}\\
    &=N\pab{V[y]+V[\tilde{y}]}-2N\cdot V[\tilde{y}]\\
    &=N\pab{V[y]-V[\tilde{y}]}
\end{align*}
となる.最小二乗推定を行った場合の$s_{e}^{2}$および$V[\tilde{y}]=V[\hat{y}]$は条件と与えられた表から求まり,
\begin{align*}
    s_{e}^{2}&=0.972\\
    V[\tilde{y}^{2}_{M}]&=\overline{\hat{y}^{2}}_{M}-\bar{y}_{M}^{2}=0.324\\
    V[\tilde{y}^{2}_{F}]&=\overline{\hat{y}^{2}}_{F}-\bar{y}_{F}^{2}=0.324
\end{align*}
となるので,$V[y_{m}]=V[y_{F}]=0.324+0.972=1.296$と求まる.\par
また,共分散$\Cov(x,y)$を考えると,$b$の最小二乗推定量が$\hat{b}=\Cov{x,y}/V[x]$と置かれていたことに注意すると,
\begin{align*}
    \Cov(x_M,y_M)&=\hat{b}_{M}\cdot V[x_{M}]=0.6\pab{\bar{x^{2}}_{M}-\bar{x}^{2}_{M}}=0.54\\
    \Cov(x_F,y_F)&=\hat{b}_{F}\cdot V[x_{F}]=0.6\pab{\bar{x^{2}}_{F}-\bar{x}^{2}_{F}}=0.54\\
\end{align*}
と求まる.よって,最小二乗推定量を考えると,$\hat{d}=\Cov\pab{x,y}/V[y]$はともに$5/12$となる.一方で,
\begin{align*}
    \hat{c}_{M}&=\bar{x}_{M}-\hat{d}\bar{y}_{M}=3.5\\
    \hat{c}_{F}&=\bar{x}_{F}-\hat{d}\bar{y}_{F}=3.0\\
\end{align*}
となる.特に同じ賃金$y=7.2$のもとでは,男性は6.5時間,女性は6.0時間の労働を行っている.
\subsection{}
前問で考察した通り.
\section{}
略.
\end{document}