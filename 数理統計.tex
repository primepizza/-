\documentclass[a4paper,11pt]{ltjsarticle}
\usepackage[no-math]{luatexja-fontspec}
\usepackage[scale=0.9]{roboto}
\usepackage{physics2}
\usepackage{derivative}
\usepackage{bm}
\usephysicsmodule{ab}
\usepackage{amsthm}
\usepackage{amsmath}
\usepackage{amssymb}
\usepackage{amsfonts}
\usepackage{mathtools}
\NewDocumentCommand{\Set}{m o O{\middle |}}{\IfValueTF{#2}{\left\{#1\ #3\ #2\right\}}{\left\{#1\right\}}}
\NewDocumentCommand{\Coex}{m m}{E\left[#1 \ \middle | #2 \right]}
\providecommand{\floor}[1]{\ab \lfloor#1\rfloor}
\newcommand{\otherwise}{\mathrm{otherwise}}
\newcommand{\suchthat}{\mathrm{\ s.t.\ }}
\newcommand{\Number}{\mathbb{N}}
\newcommand{\Zahlen}{\mathbb{Z}}
\newcommand{\Quotient}{\mathbb{Q}}
\newcommand{\Real}{\mathbb{R}}
\newcommand{\Complex}{\mathbb{C}}
\title{統計検定(統計数理)}
\author{田淵進}
\theoremstyle{definition}
\newtheorem{remark}{Remark}
\newtheorem{definition}[remark]{Definition}
\newtheorem{theorem}[remark]{Theorem}
\usepackage[japanese,english]{babel}
\begin{document}
\maketitle

\part{2023年}
1,2は簡単なので省略.問1で推定量$\alpha$の$\beta$に対する漸近相対効率を問われているが,この定義は
\begin{align}
    \lim_{n\to\infty}\frac{V[\beta]}{V[\alpha]}
\end{align}
のことである.ただし,$n$はサンプル数であり,$\alpha,\beta$は$n$に依存している値で,特に一致推定量であれば分散は$0$に収束する.
\setcounter{section}{2}
\section{}
\subsection{}
$X$は指数分布$f(x)=\lambda e^{-\lambda x}\ (x>0)$に従うので,
\begin{align*}
    E[X]&=\lambda\int^{\infty}_{0}x\lambda e^{-\lambda x}dx\\
    &= \int^{\infty}_{0} e^{-\lambda x} dx\\
    &= \frac{1}{\lambda}
\end{align*}
となる.
\subsection{}
モーメント母関数は$t<\lambda$の範囲で
\begin{align*}
    M_{X}(t)&=E[e^{tX}] \\
        &=\int^{\infty}_{0} e^{tx}\lambda e^{-\lambda x} dx \\
        &=\lambda \int^{\infty}_{0} e^{-(\lambda - t)x} dx \\
        &=\frac{\lambda}{\lambda - t}
\end{align*}
となる.
\subsection{}
確率変数$X_{W}$の従う密度は$g(x)=\frac{e^{hx}f(x)}{M_{X}(h)}$で与えられるので,$h>0$に対しては
\begin{align*}
    E[X_{W}]&=\int^{\infty}_{0} x \frac{e^{hx}\lambda e^{-\lambda x}}{M_{X}(h)} dx \\
        &=\frac{\lambda}{M_{X}(h)} \int^{\infty}_{0} x e^{-(\lambda - h)x} dx \\
        &=\frac{1}{M_{X}(h)}\int_{\infty}^{0}e^{-(\lambda - h)x} dx \\
        &=\frac{1}{M_{X}(h)} \cdot \frac{\lambda}{(\lambda - h)^{2}} \\
        &=\frac{1}{\lambda - h}\\
        &>E[X]
\end{align*}
となる.
\subsection{}
一般の確率密度$f$に従う分布$X$について,$X_{W}$のモーメント母関数を考えると,
\begin{align*}
E[e^{tX_{W}}]&=\int^{\infty}_{-\infty} e^{tx} \frac{e^{hx}f(x)}{M_{X}(h)} dx \\
    &=\frac{1}{M_{X}(h)} \int^{\infty}_{-\infty} e^{(t+h)x} f(x) dx \\
    &=\frac{M_{X}(t+h)}{M_{X}(h)}
\end{align*}
ここで,$t$に関する$r$回微分を考えると,
\begin{align*}
    E[X^{r}]&=\frac{M_{X}^{(r)}(h+0)}{M_{X}(h)}
\end{align*}
を得る.
\subsection{}
$h=0$とすると,$M_{X}(0)=1$であるので,$X_{W}$と$X$の分布は等しい.ここで,
\begin{align*}
    \odv{E[X_{W}]}{h}&=\frac{M_{X}''(h)M_{X}(h)-\pab{M_{X}'(h)}^2}{M_X(h)^{2}}\\
    &=E[X_{W}^{2}] - E[X_{W}]^{2}\\
    &=Var(X_{W})\\
    &>0
\end{align*}
となる.よって,$h>0$に対して$E[X_{W}]>E[X]$であり,$h<0$に対して$E[X_{W}]<E[X]$である.
\newpage
\section{}
\subsection{}
$W$が自由度$k$のカイ二乗分布に従うとする.このとき,
\begin{align*}
    E\bab{W}&=\frac{1}{2^{k/2}\Gamma\pab{k/2}}\int^{\infty}_{0} t\cdot t^{k/2-1} e^{-t/2} dt \\
        &=\frac{1}{2^{k/2}\Gamma\pab{k/2}} 2^{k/2+1} \Gamma\pab{\frac{k}{2}+1} \\
        &=k
\end{align*}
であり,特に$k\geq 3$においては
\begin{align*}
    E\bab{1/W}&=\int^{\infty}_{0} \frac{1}{t} \cdot \frac{1}{2^{k/2}\Gamma\pab{k/2}} t^{k/2-1} e^{-t/2} dt \\
        &=\frac{1}{2^{k/2}\Gamma\pab{k/2}} 2^{k/2-1} \Gamma\pab{\frac{k}{2}-1} \\
        &=\frac{1}{k-2}
\end{align*}
となる.
\subsection{}
今,$\mathbf{\varepsilon}\sim N(0,\sigma^{2}I_{n})$であり,係数ベクトル$\beta\in\Real^{p}$及び説明変数行列$X\in\Real^{n\times p}$について,線形回帰モデル
\begin{align*}
    Y&=X\beta+\mathbf{\varepsilon} 
\end{align*}
が与えられている.今,仮定より$X$のランクは$p$である.このとき,$X$の転置行列を$X$とすると,
\begin{align*}
    P\coloneqq X\pab{X^{T}X}^{-1}X^{T}
\end{align*}
は$\Im(X)$への直交射影である.実際,
\begin{align*}
    P^2&=X\pab{X^{T}X}^{-1}X^{T}X\pab{X^{T}X}^{-1}X^{T} \\
        &=X\pab{X^{T}X}^{-1}X^{T} \\
        &=P \\
    P^{T}&=\pab{X\pab{X^{T}X}^{-1}X^{T}}^{T} \\
        &=X\pab{X^{T}X}^{-1}X^{T} \\
        &=P
\end{align*}
が成り立つため,直交射影である.また明らかに$\Im{P}\subset \Im{X}$であり,任意の$v\in \Im{X}$に対して,ある$c\in\Real^{p}$が存在して$v=Xc$とかけるので,
\begin{align*}
    Pv&=X\pab{X^{T}X}^{-1}X^{T}v \\
        &=X\pab{X^{T}X}^{-1}X^{T}Xc \\
        &=Xc \\
        &=v\in \Im{P}
\end{align*}
となることから,$\Im{X}=\Im{P}$も成り立つ.\par
ここで,$\Im{X}$と$\Im{X}$の直交補空間の正規直交基底をそれぞれ$\{g_{1},\cdots,g_{p}\}$,$\{g_{p+1},\cdots,g_{n}\}$とする.このとき,行列
\begin{align*}
    G\coloneqq \pab{g_{1},\cdots,g_{n}}
\end{align*}
という直交行列を取ると,基底の変換によって,ある$\eta_{1},\ldots,\eta_{p}\in\Real $を用いて
\begin{align}
    G^{T}Y&=G^{T}X\beta+ G^{T}\mathbf{\varepsilon}\\
        &=\begin{pmatrix}
            \eta_{1}\\
            \vdots \\
            \eta_{p} \\
            0 \\
            \vdots \\
            0
        \end{pmatrix}
        +G^{T}\mathbf{\varepsilon}
\end{align}
と表される.特に,$G^{T}\mathbf{\varepsilon}$は再び$N(0,\sigma^{2}I_{n})$に従うので,各成分を$\gamma_{1},\ldots,\gamma_{n}$とおくと
\begin{align*}
    G^{T}\hat{Y}=X\hat{\beta}&=PY\\
        &= P\pab{X\beta+\mathbf{\varepsilon}} \\
        &=X\beta + G^{T}P\mathbf{\varepsilon} \\
        &=\begin{pmatrix}
            \eta_{1}+\gamma_{1}\\
            \vdots \\
            \eta_{p} +\gamma_{p}\\
            0 \\
            \vdots \\
            0
        \end{pmatrix}
\end{align*}
となる.よって,
\begin{align*}
    \hat{\sigma^{2}}=\frac{1}{n}\Vab{Y-X\hat{\beta}}^{2}&=\frac{1}{n}\Vab{G^{T}Y-G^{T}X\hat{\beta}}^{2} \\
        &=\frac{1}{n}\sum_{i=p+1}^{n} \gamma_{i}^{2}
\end{align*}
であり,
\begin{align*}
    \Vab{X\hat{\beta}-X\beta}^{2}&=\Vab{G^{T}X\hat{\beta}-G^{T}X\beta}^{2} \\
        &=\sum_{i=1}^{p} \gamma_{i}^{2}
\end{align*}
となる.以上から,
\begin{align*}
    W_{1}&:=n\hat{\sigma^{2}}/\sigma^{2}=\frac{1}{\sigma^{2}} \sum_{i=p+1}^{n} \gamma_{i}^{2}\\
    W_{2}&=\Vab{X\hat{\beta}-X\beta}^{2}=\frac{1}{\sigma^{2}} \sum_{i=1}^{p} \gamma_{i}^{2}
\end{align*}
はそれぞれ独立に,自由度$n-p$,$p$のカイ二乗分布に従う.
\subsection{}
前問と同じ記号を用いて,
$G^{T}\varepsilon'=(\gamma_{1}',\ldots,\gamma_{n}')^{T}$とおく.このとき,
\begin{align*}
    Z-\hat{Y}=\begin{pmatrix}
        \gamma_{1}'-\gamma_{1} \\
        \vdots \\
        \gamma_{p}'-\gamma_{p} \\
        \gamma_{p+1}' \\
        \vdots \\
        \gamma_{n}'
    \end{pmatrix}
\end{align*}
となる.よって,
\begin{align*}
    \Coex{\Vab{Z-\hat{Y}}}{Y}&=\Coex{\sum_{i=1}^{p} (\gamma_{i}'-\gamma_{i})^{2} + \sum_{i=p+1}^{n} \gamma_{i}'^{2} \mid }{Y} \\
    &= \sum_{i=1}^{p} \Coex{(\gamma_{i}'-\gamma_{i})^{2}}{Y} + \sum_{i=p+1}^{n} E\bab{\gamma_{i}'^{2}} \\
    &= n\sigma^{2} + W_{2}\sigma^{2}
\end{align*}
となる.
\subsection{}
各値がどのような確率変数かはわかっているので,計算すると
\begin{align*}
    E\bab{\frac{\Delta(Z)-\Delta(Y)}{\hat{\sigma^{2}}}}&=E\bab{\frac{n\sigma^{2}+W_{2}\sigma^{2}-W_{1}\sigma^{1}}{\sigma^{2}W_{1}/n}} \\
        &=\frac{n(n+p)}{n-p-2}-n\\
        &=\frac{2n(p+1)}{n-p-2}
\end{align*}
であり,極限は
\begin{align*}
    \lim_{n\to\infty}\frac{2n(p+1)}{n-p-2}&=2(p+1)
\end{align*}
\newpage
\section{}
\subsection{}
各$U_{i}$は有限母集団$\Set{z_{1},\ldots,z_{n}}$からの非復元抽出であるので,$E[U_{i}]=\bar{z}$であり,$V[U_{i}]=\sigma^{2}_{N}$である.よって,
\begin{align*}
    E[\bar{U}]&=\bar{z}
\end{align*}
であり,有限母集団修正を考慮すると,
\begin{align*}
    V[\bar{U}]&=\frac{\sigma^{2}_{N}}{m}\frac{N-m}{N-1}
\end{align*}
となる.
\subsection{}
いま,$m\bar{U}+n\bar{V}=(m+n)\bar{z}$であることに注意すると,
\begin{align*}
    D&=\bar{U}-\bar{V}\\
    &=\bar{U}-\frac{(m+n)\bar{z}-m\bar{U}}{n} \\
        &=\frac{m+n}{n}\pab{\bar{U}-\bar{z}}
\end{align*}
となる.また,
\begin{align*}
    (N-2)\tilde{S}^{2}=\sum^{m}_{i=1}(U_{i}-\bar{U})^2+\sum^{n}_{j=1}(V_{j}-\bar{V})^2&=\sum^{m}_{i=1}(U_{i}-\bar{z})^2+\sum^{n}_{j=1}(V_{j}-\bar{z})^2 - m(\bar{U}-\bar{z})^2 - n(\bar{V}-\bar{z})^2\\
    &=\sum^{m}_{i=1}(U_{i}-\bar{z})^2+\sum^{n}_{j=1}(V_{j}-\bar{z})^2 - \frac{mn}{m+n}(\bar{U}-\bar{V})^2\\
    &=N\sigma^{2}_{N} - \frac{mn}{m+n}D^{2}
\end{align*}
となり,
\begin{align*}
    N\sigma^{2}_{N}-(N-2)\tilde{S}^{2} = \frac{mn}{m+n}D^{2}
\end{align*}
を得る.
\subsection{}
いま,
\begin{align*}
    E[D]&=0\\
    V[D]&=\pab{\frac{m+n}{n}}^2 V[\bar{U}]\\
        &=\frac{(m+n)^{2}}{n^{2}}\cdot \frac{\sigma^{2}_{N}}{m}\cdot \frac{N-m}{N-1}\\
        &=\frac{(m+n)^{2}}{(m+n-1)mn}\sigma^{2}_{N}
\end{align*}
\subsection{}
2.の結果と3.の途中式より
\begin{align*}
    \frac{D-E[D]}{\sqrt{V[D]}}=\frac{U-E[U]}{\sqrt{V[U]}}
\end{align*}
が成り立ち,右辺は近似的に標準正規分布に従うので,左辺も近似的に標準正規分布に従う.
\subsection{}
いま,
\begin{align*}
    \tilde{T}^{2}&=\frac{D^2}{\pab{\frac{1}{m}+\frac{1}{n}}\tilde{S}^{2}}\\
    &=\frac{mn}{N}\cdot \frac{D^2}{\tilde{S}^{2}}
\end{align*}
であることに注意すると,
\begin{align*}
    \tilde{W}^{2}&=\frac{D^{2}}{V[D]}\\
    &=\frac{D^{2}(N-1)mn}{N^{2}\sigma^{2}_{N}}\\
    &=\frac{D^{2}(N-1)mn}{N(N-2)\tilde{S}^{2}+mnD^{2}}\\
    &=\frac{(N-1)mn}{N(N-2)\frac{\tilde{S}^{2}}{\tilde{D^{2}}}+ mn}\\
    &=\frac{(N-1)mn}{(N-2)\cdot \frac{mn}{\tilde{T}^{2}} + mn}\\
    &=\frac{N-1}{N-2+\tilde{T}^{2}}T^{2}
\end{align*}
よって,両辺の平方根を取れば良い.
ここで,$\tilde{W}$は近似的に標準正規分布に従うので,検定が可能である.また,関数$g$は単調増加であるため,$T$を用いても同様の検定ができる.
\end{document}