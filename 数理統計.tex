\documentclass[a4paper,11pt]{ltjsarticle}
\usepackage[no-math]{luatexja-fontspec}
\usepackage[scale=0.9]{roboto}
\usepackage{physics2}
\usepackage{derivative}
\usepackage{bm}
\usephysicsmodule{ab}
\usepackage{amsthm}
\usepackage{amsmath}
\usepackage{amssymb}
\usepackage{amsfonts}
\usepackage{mathtools}
\NewDocumentCommand{\Set}{m o O{\middle |}}{\IfValueTF{#2}{\left\{#1\ #3\ #2\right\}}{\left\{#1\right\}}}
\NewDocumentCommand{\Coex}{m m}{E\left[#1 \ \middle | #2 \right]}
\providecommand{\floor}[1]{\ab \lfloor#1\rfloor}
\newcommand{\otherwise}{\mathrm{otherwise}}
\newcommand{\suchthat}{\mathrm{\ s.t.\ }}
\newcommand{\Number}{\mathbb{N}}
\newcommand{\Zahlen}{\mathbb{Z}}
\newcommand{\Quotient}{\mathbb{Q}}
\newcommand{\Real}{\mathbb{R}}
\newcommand{\Complex}{\mathbb{C}}
\title{統計検定(統計数理)}
\author{田淵進}
\theoremstyle{definition}
\newtheorem{remark}{Remark}
\DeclareMathOperator{\Cov}{Cov}
\DeclareMathOperator{\Ker}{Ker}
\DeclareMathOperator{\Image}{Im}
\renewcommand{\Im}{\Image}
\newtheorem{definition}[remark]{Definition}
\newtheorem{theorem}[remark]{Theorem}
\usepackage[japanese,english]{babel}
\begin{document}
\maketitle
\part{2021年}
\section*{1,2,3}
略
\setcounter{section}{3}
\section{}
\subsection*{4.1, 4.2, 4.3}
\setcounter{subsection}{3}
自明な計算から
\begin{align*}
    V[\bar{X}]&=\sigma^{2}/n\\
    E[(X_{1}-X_{2})^{3}]&=0\\
    E[Y_{i}Y_{j}Y_{k}]&=\begin{cases}
        \tau&(i=j=k)\\
        0&(\otherwise)
    \end{cases}
\end{align*}
\subsection{}
\begin{align*}
    \sum^{n}_{i}(X_{i}-\bar{X})^{3}&=\sum^{n}_{i}(Y_{i}-\bar{X})^{3}\\
    &=\sum^{n}_{i=1}\pab{Y_{i}^{3}-3Y_{i}^{2}\bar{Y}+3Y_{i}\bar{Y}^{2}-\bar{Y}^{3}}
\end{align*}
となるので,期待値を各項で計算して
\begin{align*}
   E\bab{\sum^{n}_{i}(X_{i}-\bar{X})^{3}} &=n\tau-3n\cdot \frac{\tau}{n} + 3n\cdot\frac{\tau}{n^{2}}+n\cdot\frac{n\cdot\tau}{n^{3}}\\
    &=\tau\pab{n-3+\frac{2}{m}}
\end{align*}
\subsection{}
まず,
\begin{align*}
    \hat\tau&=\pab{\sum_{i=1}^{n}a_{i}\pab{Y_{i}+\mu}}^{3}\\
            &=\pab{\pab{\sum_{i=1}^{n}a_{i}Y_{i}}+\pab{\sum_{i=1}^{n}a_{i}\mu}}^{3}\\
            &=\pab{\sum_{i=1}^{n}a_{i}Y_{i}}^{3}+3\pab{\sum_{i=1}^{n}a_{i}Y_{i}}^{2}\pab{\sum_{i=1}^{n}a_{i}\mu}+3\pab{\sum_{i=1}^{n}a_{i}Y_{i}}\pab{\sum_{i=1}^{n}a_{i}\mu}^{2}+\pab{\sum_{i=1}^{n}a_{i}\mu}^{3}
\end{align*}
この期待値を取ると
\begin{align*}
    \tau\pab{\sum_{i=1}^{n}a_{i}^{3}}+3\mu\sigma^{2}\pab{\sum_{i=1}^{n}a_{i}^{2}}\pab{\sum_{i=1}^{n}a_{i}}+\mu\pab{\sum_{i=1}^{n}a_{i}}^{3}
\end{align*}
となるので,求める条件は
\begin{align*}
    \sum_{i=1}^{n}a_{i}&=0\\
    \sum_{i=1}^{n}a_{i}^{3}&=\tau
\end{align*}
となる.
\newpage
\section{}
\subsection{}
\begin{align*}
    V[y]=E[(Lx-L\mu)^{T}(Lx-L\mu)]&=L^{T}V[x]L\\
    &=L^{T}L
\end{align*}
\subsection{}
共分散が$0$であれば良いので,
\begin{align*}
    \Cov\pab{Lx,Mx}=L^{T}M=O
\end{align*}
が求める条件である.
\subsection{}
\subsection{}
一般に,$L,M$に対して,
\begin{align*}
    LM^{T}=AMM^{T}
\end{align*}
が成り立つような$l\times m$行列$A$が存在するための条件を考える.両辺を転置して
\begin{align}
    \label{eq:exist-A}
    ML^{T}=MM^{T}A^{T}
\end{align}
となる.ここで,一般に$\Im{M}=\Im{MM^{T}}$が成り立つことに注意すると,$\Im{ML^{T}}\subset \Im{MM^{T}}$となるので,任意の$u\in \Im{ML^{T}}\subset$に対して,$v\in \Real^{m}$を取って
\begin{align*}
    u=(MM^{T})v
\end{align*}
となる$v$が存在する.特に,$r=rank(M)$として,一次独立な$x_{1},\ldots,x_{r}\in \Real^{l}$に対して,$Mx_{i}=(MM^{T})v_{i}$となる$v_{i}\in \Real^{m}$が存在するので,$A^{T}$を$A^{T}x_{i}=v_{i}$かつその他の空間で$0$となる写像として特徴づければ,$A$は常に存在することがわかる.\par
特に$M$がフルランクであれば,$MM^{T}$が可逆になるので,
\begin{align*}
    A=(MM^{T})^{-1}LM^{T}
\end{align*}
として取ることができる.

\part{2022年}
\setcounter{section}{0}
\section{}
\subsection{}
$A,B,C$がすべて独立のとき,
\begin{align*}
    P(A\cap B)&=P(A)P(B)=\frac{9}{16}\\
    P(A\cap B \cap C)&=P(A)P(B)P(C)=\frac{27}{64}
\end{align*}
である.
\subsection{}
上界は$3/4$であり,下界は余事象を考えて$1/2$である.よって$P(A\cap B)$の取りうる値を$I$とすると,$I\subset [1/2,3/4]$である.逆の包含を構成的に示す.確率空間として$\Omega=[0,1]$にLebesgue測度を入れたものを考える.このとき,$r\in [0,1/4]$として$A=[0,3/4],B=[r,r+3/4]$とおくと,$P(A)=P(B)=3/4$および$A\cap B=[r,3/4]$であり,この確率は$[1/2,3/4]$の範囲を動く.\par
よって$I=[1/2,3/4]$
\subsection{}
$P(A\cap B\cap C)$の取りうる範囲を$J$とすると,$A=C$の場合を考えると$J\supset I$がわかる.また,上界は明らかに$3/4$である.一方,下界は余事象を考えて$1/4$であるので,$[1/4,1/2]$の範囲を$P(A\cap B \cap C)$が取りうることを示す.確率空間として前問と同じ$\Omega,A,B$を$r=1/4$として取ったうえで,$C=[0,p]\cup[3/4,3/2-p]\ (0\in[1/2,3/4])$とおくと,$P(C)=3/4$かつ$A\cap B\cap C=[1/4,p]$であり,この確率は$[1/4,1/2]$の範囲を動く.\par
よって$J=[1/4,3/4]$
\subsection{}
いま,有限な確率空間として
\begin{align*}
    \Omega&=\Set{\varepsilon,a,b,ab, c, ac, bc, abc}\\
    A&=\Set{a,ab,ac,abc}\\ 
    B&=\Set{b,ab,bc,abc}\\
    C&=\Set{c,ac,bc,abc}
\end{align*}
と置くことができる.このとき,条件から$P(A\cap B^{c})=3/16$などが成り立つので,対称性から
\begin{align*}
    P(\varepsilon)&=s-1/8\\
    P(a)&=3/16-s\\
    P(b)&=3/16-s\\
    P(c)&=3/16-s\\
    P(ab)&=s\\
    P(bc)&=s\\
    P(ac)&=s\\
    P(abc)&=9/16-s
\end{align*}
とかける.それぞれが非負になるための条件は$1/8\leq s \leq 3/16$である.よって$P(A\cap B\cap C)=9/16-s$の取りうる値は$[3/8,7/16]$である.
\newpage
\section{}
\subsection{}
定義から
\begin{align*}
    F_{1}(u)=P(U\leq u)&=\frac{1+u}{2}\\
    F_{2}(v)=P(V\leq v)&=\frac{1+v}{2}
\end{align*}
すなわち,$[-1,1]$で一様分布に従う.
\subsection{}
$(U,V)$の同時密度は,
\begin{align*}
    \pdv{F(u,v)}{u}{v}&=1/4
\end{align*}
となるので,$(U,V)$は$[-1,1]\times[-1,1]$で一様分布に従い,成分ごとに独立である.
\subsection{}
座標平面上に$u^{2}+v^{2}\leq 1$となる領域を描くと,この面積の$1/4$倍が確率に等しいので,$P(U^{2}+V^{2}\leq 1)=\pi/4$となる.
\subsection{}
同様に,$\vab{u-v}\leq 1$の領域を考えて
\begin{align*}
    P(\vab{U-V}\leq 1)&=\frac{3}{4}
\end{align*}
である.
\subsection{}
$U^{2}-2UV+V^{2}\leq 1$の条件のもとで,$(u,v)$はこの領域を一様に分布する.よって
\begin{align*}
    E[V]&=\frac{1}{3}\int^{0}_{-1}v\int^{v+1}_{-1} du dv + \frac{1}{3}\int^{1}_{0} v \int^{1}_{v-1} du dv \\
        &=\frac{1}{3}\int^{0}_{-1} v(v+2) dv + \frac{1}{3}\int^{1}_{0} v(2-v) dv \\
        &=0 \\
    E[V^2]&=\frac{1}{3}\int^{0}_{-1} v^{2}\int^{v+1}_{-1} du dv + \frac{1}{3}\int^{1}_{0} v^{2} \int^{1}_{v-1} du dv \\
        &=\frac{1}{3}\int^{0}_{-1} v^{2}(v+2) dv + \frac{1}{3}\int^{1}_{0} v^{2}(2-v) dv \\
        &=\frac{5}{18}
\end{align*}
となる.対称性から$E[U]=0$,$E[U^{2}]=5/18$であるので,共分散は
\begin{align*}
    Cov(U,V)&=E[UV]-E[U]E[V]\\
        &=E[UV]\\
        &=\frac{1}{3}\int^{0}_{-1} v \int^{v+1}_{-1} u du dv + \frac{1}{3}\int^{1}_{0} v \int^{1}_{v-1} u du dv \\
        &=\frac{1}{3}\int^{0}_{-1} v \cdot \frac{(v+1)^{2}-1}{2} dv + \frac{1}{3}\int^{1}_{0} v \cdot \frac{1-(v-1)^{2}}{2} dv \\
        &= \frac{5}{36}
\end{align*}
よって,相関係数は$1/2$となる.
\newpage
\section{}
\subsection{}
$X$はパラメータ$\lambda$のPouisson分布に従うので,
\begin{align*}
    E[X]&=\lambda\\
    V[X]&=\lambda
\end{align*}
である.
\subsection{}
$\Lambda$はパラメータ$\alpha,\beta$のGamma分布に従うので,
\begin{align*}
    E[\Lambda]&=\frac{\beta^\alpha}{\Gamma(\alpha)} \int^{\infty}_{0} \lambda \cdot \lambda^{\alpha -1} e^{-\beta \lambda} d\lambda \\
        &=\frac{\beta^\alpha}{\Gamma(\alpha)} \frac{\Gamma(\alpha+1)}{\beta^{\alpha +1}}\\
        &=\frac{\alpha}{\beta}\\
    E[\Lambda^{2}]&=\frac{\beta^\alpha}{\Gamma(\alpha)} \int^{\infty}_{0} \lambda^{2} \cdot \lambda^{\alpha -1} e^{-\beta \lambda} d\lambda \\
        &=\frac{\beta^\alpha}{\Gamma(\alpha)} \frac{\Gamma(\alpha+2)}{\beta^{\alpha +2}}\\
        &=\frac{(\alpha +1)\alpha}{\beta^{2}}
\end{align*}
ゆえ,
\begin{align*}
    V[\Lambda]&=E[\Lambda^{2}]-E[\Lambda]^{2}\\
        &=\frac{\alpha}{\beta^{2}}
\end{align*}
となる.
\subsection{}
$X$の分布は,
\begin{align*}
    P(X=k)&=\int^{\infty}_{0} P(X=k \mid \Lambda=\lambda) f_{\Lambda}(\lambda) d\lambda \\
        &=\int^{\infty}_{0} \frac{\lambda^{k} e^{-\lambda}}{k!} \cdot \frac{\beta^{\alpha}}{\Gamma(\alpha)} \lambda^{\alpha -1} e^{-\beta \lambda} d\lambda \\
        &=\frac{\Gamma(\alpha+k)}{\Gamma(k+1)\Gamma(\alpha)} \cdot \pab{\frac{1}{1+\beta}}^{k} \cdot \pab{\frac{\beta}{1+\beta}}^{\alpha}
\end{align*}
と求まる.
\subsection{}
いま,$X$のモーメント母関数を考えると
\begin{align*}
    M_{X}(t)&=E[e^{tX}]\\
        &=\sum^{\infty}_{k=0}  \frac{\Gamma(\alpha+k)}{\Gamma(k+1)\Gamma(\alpha)} \cdot \pab{\frac{e^t}{1+\beta}}^{k} \cdot \pab{\frac{\beta}{1+\beta}}^{\alpha}  
\end{align*}
ここで,$\gamma+1=e^{-t}(\beta+1)$とおくと,
\begin{align*}
    M_{X}(t)&=\pab{\frac{\beta}{\beta + 1}}^{\alpha}\cdot \pab{\frac{\gamma+1}{\gamma}}^{\alpha} \sum^{\infty}_{k=0} \frac{\Gamma(\alpha+k)}{\Gamma(k+1)\Gamma(\alpha)} \cdot\pab{\frac{\gamma}{\gamma+1}}^{\alpha} \pab{\frac{1}{1+\gamma}}^{k} \\
    &=\pab{\frac{\beta}{e^{t}\gamma}}^{\alpha}\\
    &=\pab{\frac{\beta}{1+\beta - e^{t}}}^{\alpha}
\end{align*}
となる.$t$について微分すると
\begin{align*}
    M_{X}'(t)&=\alpha \pab{\frac{\beta}{1+\beta - e^{t}}}^{\alpha +1} e^{t}\\
    M_{X}''(t)&=\alpha \pab{\frac{\beta}{1+\beta - e^{t}}}^{\alpha +1} e^{t} + \alpha(\alpha +1) \pab{\frac{\beta}{1+\beta - e^{t}}}^{\alpha +2} e^{2t}
\end{align*}
であるので,
\begin{align*}
    E[X]&=M_{X}'(0)=\frac{\alpha}{\beta}\\
    V[X]&=M_{X}''(0)-E[X]^{2}=\frac{\alpha}{\beta} + \frac{\alpha}{\beta^{2}}
\end{align*}
を得る.
\subsection{}
モーメント法においては,推定量は
\begin{align*}
    \frac{\tilde{\alpha}}{\tilde{\beta}}=\bar{X}\\
    \frac{\tilde{\alpha}}{\tilde{\beta}} + \frac{\tilde{\alpha}}{\tilde{\beta}^{2}}=\tilde{S}^{2}
\end{align*}
となる.
\begin{align*}
    \hat{\alpha}&=\frac{\bar{X}^{2}}{S^{2}-\bar{X}}\\
    \hat{\beta}&=\frac{\bar{X}}{S^{2}-\bar{X}}
\end{align*}
を得る.特に,$S^{2}>\bar{X}$のときに推定量は正になる.
\newpage
\section{}
略
\section{}
\subsection{}
正規分布の再生性から,
\begin{align*}
    D_{i}&=Y_{i}-X_{i} \sim N(\theta,2\sigma^{2})
\end{align*}
となる.
\subsection{}
次のように取れば良い.
\begin{align*}
    \nu&=\frac{\theta}{2}+\frac{1}{n}\sum^{n}_{i=1} \mu\\
    a_{i}&=\mu_{i}-\frac{1}{n}\sum^{n}_{i=1} \mu_{i}\\
    b_{1}&=-\theta/2\\
    b_{2}&=\theta/2
\end{align*}
実際,このようにおくと,
\begin{align*}
    Z_{i1}=\mu_{i}+\varepsilon_{i1}\\
    Z_{i2}=\mu_{i}+\theta + \varepsilon_{i2}
\end{align*}
となる.
\subsection{}
いま,
\begin{align*}
    \bar{Z}_{\cdot\cdot}&\coloneqq \frac{1}{2n} \sum^{n}_{i=1} (Z_{i1}+Z_{i2})\\
    \bar{Z}_{i\cdot}&\coloneqq \frac{1}{2}(Z_{i1}+Z_{i2})\\
    \bar{Z}_{\cdot j}&\coloneqq \frac{1}{n} \sum^{n}_{i=1} Z_{ij}
\end{align*}
と定めると,
\begin{align*}
    S_{A}&= \sum^{2}_{j=1}\sum^{n}_{i=1} (\bar{Z}_{i\cdot} - \bar{Z}_{\cdot\cdot})^{2}\\
    S_{B}&= \sum^{n}_{i=1}\sum^{2}_{j=1} (Z_{ij} - \bar{Z}_{\cdot j})^{2}\\
    S_{E}&=\sum^{n}_{i=1} \sum^{2}_{j=1} (Z_{ij} - \bar{Z}_{i\cdot}-\bar{Z}_{\cdot j} + \bar{Z}_{\cdot\cdot})^{2}
\end{align*}
このとき,$S_{A}^{2}/\sigma^{2}$は自由度$\phi_A$の非心$\chi^{2}$分布に従い,$S_{B}^{2}\sigma^{2}$は自由度$\phi_B$の非心$\chi^{2}$分布に従い,$S_{E}^{2}/\sigma^{2}$は自由度$\phi_E$の$\chi^{2}$分布に従う.また,このとき,$S_{A},S_{B},S_{E}$は独立である.
よって,仮説$\theta=0$のもとでは$S_{B}$が中心$\chi^{2}$分布に従い,
\begin{align*}
    \frac{S^{2}_{A}\phi_{E}}{S^{2}_{E}\phi_{A}}&\sim F(\phi_{A},\phi_{E})
\end{align*}
となり,仮説$\mu_{i}\equiv\mu$のもとでは$S_{A}$が中心$\chi^{2}$分布に従い,
\begin{align*}
    \frac{S^{2}_{B}\phi_{E}}{S^{2}_{E}\phi_{B}}&\sim F(\phi_{B},\phi_{E})
\end{align*}
となる
\subsection{}
$\theta$に関する検定において,$X_{1}$を欠測した場合,$Y_{1}$の値からは$\theta,\mu_{i}$の寄与が分離できないため,$Y_{1}$を利用して検定を行うことはできない.\par
実際,$S_{A}$の定義の和の一行目が$0$となっており,残りの$n-1$個の和の形になっている.
\subsection{}
$\mu_{i}$に関する検定において,$X_{1}$を欠測した場合でも,$Y_{1}$と他の$Y_{i}$の値から,$\mu_1$の大きさに関する検定ができる.
\par
実際,$S_{B}$の定義の和の一行目は$Y_{1}$を含んでおり,他の$Y_{i}$とも比較できる形になっている.
\newpage
\part{2023年}
1,2は簡単なので省略.問1で推定量$\alpha$の$\beta$に対する漸近相対効率を問われているが,この定義は
\begin{align}
    \lim_{n\to\infty}\frac{V[\beta]}{V[\alpha]}
\end{align}
のことである.ただし,$n$はサンプル数であり,$\alpha,\beta$は$n$に依存している値で,特に一致推定量であれば分散は$0$に収束する.
\setcounter{section}{2}
\section{}
\subsection{}
$X$は指数分布$f(x)=\lambda e^{-\lambda x}\ (x>0)$に従うので,
\begin{align*}
    E[X]&=\lambda\int^{\infty}_{0}x\lambda e^{-\lambda x}dx\\
    &= \int^{\infty}_{0} e^{-\lambda x} dx\\
    &= \frac{1}{\lambda}
\end{align*}
となる.
\subsection{}
モーメント母関数は$t<\lambda$の範囲で
\begin{align*}
    M_{X}(t)&=E[e^{tX}] \\
        &=\int^{\infty}_{0} e^{tx}\lambda e^{-\lambda x} dx \\
        &=\lambda \int^{\infty}_{0} e^{-(\lambda - t)x} dx \\
        &=\frac{\lambda}{\lambda - t}
\end{align*}
となる.
\subsection{}
確率変数$X_{W}$の従う密度は$g(x)=\frac{e^{hx}f(x)}{M_{X}(h)}$で与えられるので,$h>0$に対しては
\begin{align*}
    E[X_{W}]&=\int^{\infty}_{0} x \frac{e^{hx}\lambda e^{-\lambda x}}{M_{X}(h)} dx \\
        &=\frac{\lambda}{M_{X}(h)} \int^{\infty}_{0} x e^{-(\lambda - h)x} dx \\
        &=\frac{1}{M_{X}(h)}\int_{\infty}^{0}e^{-(\lambda - h)x} dx \\
        &=\frac{1}{M_{X}(h)} \cdot \frac{\lambda}{(\lambda - h)^{2}} \\
        &=\frac{1}{\lambda - h}\\
        &>E[X]
\end{align*}
となる.
\subsection{}
一般の確率密度$f$に従う分布$X$について,$X_{W}$のモーメント母関数を考えると,
\begin{align*}
E[e^{tX_{W}}]&=\int^{\infty}_{-\infty} e^{tx} \frac{e^{hx}f(x)}{M_{X}(h)} dx \\
    &=\frac{1}{M_{X}(h)} \int^{\infty}_{-\infty} e^{(t+h)x} f(x) dx \\
    &=\frac{M_{X}(t+h)}{M_{X}(h)}
\end{align*}
ここで,$t$に関する$r$回微分を考えると,
\begin{align*}
    E[X^{r}]&=\frac{M_{X}^{(r)}(h+0)}{M_{X}(h)}
\end{align*}
を得る.
\subsection{}
$h=0$とすると,$M_{X}(0)=1$であるので,$X_{W}$と$X$の分布は等しい.ここで,
\begin{align*}
    \odv{E[X_{W}]}{h}&=\frac{M_{X}''(h)M_{X}(h)-\pab{M_{X}'(h)}^2}{M_X(h)^{2}}\\
    &=E[X_{W}^{2}] - E[X_{W}]^{2}\\
    &=Var(X_{W})\\
    &>0
\end{align*}
となる.よって,$h>0$に対して$E[X_{W}]>E[X]$であり,$h<0$に対して$E[X_{W}]<E[X]$である.
\newpage
\section{}
\subsection{}
$W$が自由度$k$のカイ二乗分布に従うとする.このとき,
\begin{align*}
    E\bab{W}&=\frac{1}{2^{k/2}\Gamma\pab{k/2}}\int^{\infty}_{0} t\cdot t^{k/2-1} e^{-t/2} dt \\
        &=\frac{1}{2^{k/2}\Gamma\pab{k/2}} 2^{k/2+1} \Gamma\pab{\frac{k}{2}+1} \\
        &=k
\end{align*}
であり,特に$k\geq 3$においては
\begin{align*}
    E\bab{1/W}&=\int^{\infty}_{0} \frac{1}{t} \cdot \frac{1}{2^{k/2}\Gamma\pab{k/2}} t^{k/2-1} e^{-t/2} dt \\
        &=\frac{1}{2^{k/2}\Gamma\pab{k/2}} 2^{k/2-1} \Gamma\pab{\frac{k}{2}-1} \\
        &=\frac{1}{k-2}
\end{align*}
となる.
\subsection{}
今,$\mathbf{\varepsilon}\sim N(0,\sigma^{2}I_{n})$であり,係数ベクトル$\beta\in\Real^{p}$及び説明変数行列$X\in\Real^{n\times p}$について,線形回帰モデル
\begin{align*}
    Y&=X\beta+\mathbf{\varepsilon} 
\end{align*}
が与えられている.今,仮定より$X$のランクは$p$である.このとき,$X$の転置行列を$X$とすると,
\begin{align*}
    P\coloneqq X\pab{X^{T}X}^{-1}X^{T}
\end{align*}
は$\Im(X)$への直交射影である.実際,
\begin{align*}
    P^2&=X\pab{X^{T}X}^{-1}X^{T}X\pab{X^{T}X}^{-1}X^{T} \\
        &=X\pab{X^{T}X}^{-1}X^{T} \\
        &=P \\
    P^{T}&=\pab{X\pab{X^{T}X}^{-1}X^{T}}^{T} \\
        &=X\pab{X^{T}X}^{-1}X^{T} \\
        &=P
\end{align*}
が成り立つため,直交射影である.また明らかに$\Im{P}\subset \Im{X}$であり,任意の$v\in \Im{X}$に対して,ある$c\in\Real^{p}$が存在して$v=Xc$とかけるので,
\begin{align*}
    Pv&=X\pab{X^{T}X}^{-1}X^{T}v \\
        &=X\pab{X^{T}X}^{-1}X^{T}Xc \\
        &=Xc \\
        &=v\in \Im{P}
\end{align*}
となることから,$\Im{X}=\Im{P}$も成り立つ.\par
ここで,$\Im{X}$と$\Im{X}$の直交補空間の正規直交基底をそれぞれ$\{g_{1},\cdots,g_{p}\}$,$\{g_{p+1},\cdots,g_{n}\}$とする.このとき,行列
\begin{align*}
    G\coloneqq \pab{g_{1},\cdots,g_{n}}
\end{align*}
という直交行列を取ると,基底の変換によって,ある$\eta_{1},\ldots,\eta_{p}\in\Real $を用いて
\begin{align}
    G^{T}Y&=G^{T}X\beta+ G^{T}\mathbf{\varepsilon}\\
        &=\begin{pmatrix}
            \eta_{1}\\
            \vdots \\
            \eta_{p} \\
            0 \\
            \vdots \\
            0
        \end{pmatrix}
        +G^{T}\mathbf{\varepsilon}
\end{align}
と表される.特に,$G^{T}\mathbf{\varepsilon}$は再び$N(0,\sigma^{2}I_{n})$に従うので,各成分を$\gamma_{1},\ldots,\gamma_{n}$とおくと
\begin{align*}
    G^{T}\hat{Y}=X\hat{\beta}&=PY\\
        &= P\pab{X\beta+\mathbf{\varepsilon}} \\
        &=X\beta + G^{T}P\mathbf{\varepsilon} \\
        &=\begin{pmatrix}
            \eta_{1}+\gamma_{1}\\
            \vdots \\
            \eta_{p} +\gamma_{p}\\
            0 \\
            \vdots \\
            0
        \end{pmatrix}
\end{align*}
となる.よって,
\begin{align*}
    \hat{\sigma^{2}}=\frac{1}{n}\Vab{Y-X\hat{\beta}}^{2}&=\frac{1}{n}\Vab{G^{T}Y-G^{T}X\hat{\beta}}^{2} \\
        &=\frac{1}{n}\sum_{i=p+1}^{n} \gamma_{i}^{2}
\end{align*}
であり,
\begin{align*}
    \Vab{X\hat{\beta}-X\beta}^{2}&=\Vab{G^{T}X\hat{\beta}-G^{T}X\beta}^{2} \\
        &=\sum_{i=1}^{p} \gamma_{i}^{2}
\end{align*}
となる.以上から,
\begin{align*}
    W_{1}&:=n\hat{\sigma^{2}}/\sigma^{2}=\frac{1}{\sigma^{2}} \sum_{i=p+1}^{n} \gamma_{i}^{2}\\
    W_{2}&=\Vab{X\hat{\beta}-X\beta}^{2}=\frac{1}{\sigma^{2}} \sum_{i=1}^{p} \gamma_{i}^{2}
\end{align*}
はそれぞれ独立に,自由度$n-p$,$p$のカイ二乗分布に従う.
\subsection{}
前問と同じ記号を用いて,
$G^{T}\varepsilon'=(\gamma_{1}',\ldots,\gamma_{n}')^{T}$とおく.このとき,
\begin{align*}
    Z-\hat{Y}=\begin{pmatrix}
        \gamma_{1}'-\gamma_{1} \\
        \vdots \\
        \gamma_{p}'-\gamma_{p} \\
        \gamma_{p+1}' \\
        \vdots \\
        \gamma_{n}'
    \end{pmatrix}
\end{align*}
となる.よって,
\begin{align*}
    \Coex{\Vab{Z-\hat{Y}}}{Y}&=\Coex{\sum_{i=1}^{p} (\gamma_{i}'-\gamma_{i})^{2} + \sum_{i=p+1}^{n} \gamma_{i}'^{2} \mid }{Y} \\
    &= \sum_{i=1}^{p} \Coex{(\gamma_{i}'-\gamma_{i})^{2}}{Y} + \sum_{i=p+1}^{n} E\bab{\gamma_{i}'^{2}} \\
    &= n\sigma^{2} + W_{2}\sigma^{2}
\end{align*}
となる.
\subsection{}
各値がどのような確率変数かはわかっているので,計算すると
\begin{align*}
    E\bab{\frac{\Delta(Z)-\Delta(Y)}{\hat{\sigma^{2}}}}&=E\bab{\frac{n\sigma^{2}+W_{2}\sigma^{2}-W_{1}\sigma^{1}}{\sigma^{2}W_{1}/n}} \\
        &=\frac{n(n+p)}{n-p-2}-n\\
        &=\frac{2n(p+1)}{n-p-2}
\end{align*}
であり,極限は
\begin{align*}
    \lim_{n\to\infty}\frac{2n(p+1)}{n-p-2}&=2(p+1)
\end{align*}
\newpage
\section{}
\subsection{}
各$U_{i}$は有限母集団$\Set{z_{1},\ldots,z_{n}}$からの非復元抽出であるので,$E[U_{i}]=\bar{z}$であり,$V[U_{i}]=\sigma^{2}_{N}$である.よって,
\begin{align*}
    E[\bar{U}]&=\bar{z}
\end{align*}
であり,有限母集団修正を考慮すると,
\begin{align*}
    V[\bar{U}]&=\frac{\sigma^{2}_{N}}{m}\frac{N-m}{N-1}
\end{align*}
となる.
\subsection{}
いま,$m\bar{U}+n\bar{V}=(m+n)\bar{z}$であることに注意すると,
\begin{align*}
    D&=\bar{U}-\bar{V}\\
    &=\bar{U}-\frac{(m+n)\bar{z}-m\bar{U}}{n} \\
        &=\frac{m+n}{n}\pab{\bar{U}-\bar{z}}
\end{align*}
となる.また,
\begin{align*}
    (N-2)\tilde{S}^{2}=\sum^{m}_{i=1}(U_{i}-\bar{U})^2+\sum^{n}_{j=1}(V_{j}-\bar{V})^2&=\sum^{m}_{i=1}(U_{i}-\bar{z})^2+\sum^{n}_{j=1}(V_{j}-\bar{z})^2 - m(\bar{U}-\bar{z})^2 - n(\bar{V}-\bar{z})^2\\
    &=\sum^{m}_{i=1}(U_{i}-\bar{z})^2+\sum^{n}_{j=1}(V_{j}-\bar{z})^2 - \frac{mn}{m+n}(\bar{U}-\bar{V})^2\\
    &=N\sigma^{2}_{N} - \frac{mn}{m+n}D^{2}
\end{align*}
となり,
\begin{align*}
    N\sigma^{2}_{N}-(N-2)\tilde{S}^{2} = \frac{mn}{m+n}D^{2}
\end{align*}
を得る.
\subsection{}
いま,
\begin{align*}
    E[D]&=0\\
    V[D]&=\pab{\frac{m+n}{n}}^2 V[\bar{U}]\\
        &=\frac{(m+n)^{2}}{n^{2}}\cdot \frac{\sigma^{2}_{N}}{m}\cdot \frac{N-m}{N-1}\\
        &=\frac{(m+n)^{2}}{(m+n-1)mn}\sigma^{2}_{N}
\end{align*}
\subsection{}
2.の結果と3.の途中式より
\begin{align*}
    \frac{D-E[D]}{\sqrt{V[D]}}=\frac{U-E[U]}{\sqrt{V[U]}}
\end{align*}
が成り立ち,右辺は近似的に標準正規分布に従うので,左辺も近似的に標準正規分布に従う.
\subsection{}
いま,
\begin{align*}
    \tilde{T}^{2}&=\frac{D^2}{\pab{\frac{1}{m}+\frac{1}{n}}\tilde{S}^{2}}\\
    &=\frac{mn}{N}\cdot \frac{D^2}{\tilde{S}^{2}}
\end{align*}
であることに注意すると,
\begin{align*}
    \tilde{W}^{2}&=\frac{D^{2}}{V[D]}\\
    &=\frac{D^{2}(N-1)mn}{N^{2}\sigma^{2}_{N}}\\
    &=\frac{D^{2}(N-1)mn}{N(N-2)\tilde{S}^{2}+mnD^{2}}\\
    &=\frac{(N-1)mn}{N(N-2)\frac{\tilde{S}^{2}}{\tilde{D^{2}}}+ mn}\\
    &=\frac{(N-1)mn}{(N-2)\cdot \frac{mn}{\tilde{T}^{2}} + mn}\\
    &=\frac{N-1}{N-2+\tilde{T}^{2}}\tilde{T}^{2}
\end{align*}
よって,両辺の平方根を取れば良い.
ここで,$\tilde{W}$は近似的に標準正規分布に従うので,検定が可能である.また,関数$g$は単調増加であるため,$T$を用いても同様の検定ができる.
\end{document}